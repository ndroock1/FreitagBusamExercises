\section{TMA: Some Elementary Theorems on the Distribution of Prime Numbers}

\subsection[Problem 1]{Problem 1: Apo 4.1}
Let  $S = \{1, 5, 9, 13, 17, \ldots \}$ denote the set of all positive integers of the form $4n + 1$.
An element $p$ of $S$  is called an S-prime if $p > 1$ and if the only divisors of p, among the elements of S,
are $1$ and $p$.  (For example, $49$ is an S-prime .) An element $n > 1$ in S which is not an S-prime is called an
S-composite.
\begin{itemize}
    \item[a)] Prove that every S-composite is a product of S-primes.
    \item[b)] Find the smallest S-composite that can be expressed in  more than one way as a product of S-primes.
\end{itemize}

\subsection[Problem 2]{Problem 2: Apo 4.2}
Consider $T = \{1, 7, 11, 13, 17, 19, 23, 29\}$.
\begin{itemize}
    \item[a)] For each prime $p$ in the interval $30 < p < 100$ determine a pair of integers $m, n$, where $m > 0$
    and $n \in T$, such that $p = 30m +  n$.
    \item[b)] Prove or disprove: "Every prime $p > 5$ can be expressed in the form $30m + n$, $m, n$, where $m > 0$
    and $n \in T$".
\end{itemize}

\subsection[Problem 3]{Problem 3: Apo 4.3}
.

\subsection[Problem 4]{Problem 4: Apo 4.4}

\subsection[Problem 5]{Problem 5: Apo 4.5}

\subsection[Problem 6]{Problem 6: Apo 4.6}

\subsection[Problem 7]{Problem 7: Apo 4.7}

\subsection[Problem 8]{Problem 8: Apo 4.8}

\subsection[Problem 9]{Problem 9: Apo 4.9}

\subsection[Problem 10]{Problem 10: Apo 4.10}

\subsection[Problem 11]{Problem 11: Apo 4.11}

\subsection[Problem 12]{Problem 12: Apo 4.12}

\subsection[Problem 13]{Problem 13: Apo 4.13}

\subsection[Problem 14]{Problem 14: Apo 4.14}

\subsection[Problem 15]{Problem 10: Apo 4.15}

\subsection[Problem 16]{Problem 10: Apo 4.16}

\subsection[Problem 17]{Problem 10: Apo 4.17}

\subsection[Problem 18]{Problem 10: Apo 4.18}

\subsection[Problem 19]{Problem 10: Apo 4.19}

\subsection[Problem 20]{Problem 10: Apo 4.20}

\subsection[Problem 21]{Problem 10: Apo 4.21}

\subsection[Problem 22]{Problem 10: Apo 4.22}

\subsection[Problem 23]{Problem 10: Apo 4.23}

\subsection[Problem 24]{Problem 10: Apo 4.24}

\subsection[Problem 25]{Problem 10: Apo 4.25}

\subsection[Problem 26]{Problem 10: Apo 4.26}

\subsection[Problem 27]{Problem 10: Apo 4.27}

\subsection[Problem 28]{Problem 10: Apo 4.28}

\subsection[Problem 29]{Problem 10: Apo 4.29}

\subsection[Problem 30]{Problem 10: Apo 4.30}