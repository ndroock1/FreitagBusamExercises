\section{Exercises}

\subsection{Question 1 a).}
\noindent
Find all elements $A$ of $\Gamma$ which commute with $S$.

\subsection*{Answer}
\noindent
We search for a matrix $A$, such that $A.S = S.A \wedge A \in \Gamma$.
\begin{align*}
    \begin{pmatrix}
        a & b\\
        c & d
    \end{pmatrix} \cdot
    \begin{pmatrix}
        0 & -1\\
        1 & 0
    \end{pmatrix}
    &=
    \begin{pmatrix}
        0 & -1\\
        1 & 0
    \end{pmatrix} \cdot
    \begin{pmatrix}
        a & b\\
        c & d
    \end{pmatrix} \\
    \begin{pmatrix}
        b & -a\\
        d & -c
    \end{pmatrix}
    &=
    \begin{pmatrix}
        -c & -d\\
        a & b
    \end{pmatrix}
\end{align*}
Clearly, the elements which commute are:
\[
    \begin{pmatrix}
        a & -c\\
        c & a
    \end{pmatrix}
    \wedge a^2 + c^2 = 1 \wedge a,c \in \mathbb{Z}.
\]
This leaves the following commuting matrices.
\[
    \begin{pmatrix}
        1 & 0\\
        0 & 1
    \end{pmatrix},
    \begin{pmatrix}
        -1 & 0\\
        0 & -1
    \end{pmatrix},
    \begin{pmatrix}
        0 & -1\\
        1 & 0
    \end{pmatrix},
    \begin{pmatrix}
        0 & 1\\
        -1 & 0
    \end{pmatrix}.
\]



\subsection{Question 1 b).}
\noindent
Find all elements $A$ of $\Gamma$ which commute with $ST$.

\subsection*{Answer}
\noindent
We search for a matrix $A$, such that $A.(ST) = (ST).A \wedge A \in \Gamma$.
\begin{align*}
    \begin{pmatrix}
        a & b\\
        c & d
    \end{pmatrix} \cdot
    \begin{pmatrix}
        0 & -1\\
        1 & 1
    \end{pmatrix}
    &=
    \begin{pmatrix}
        0 & -1\\
        1 & 1
    \end{pmatrix} \cdot
    \begin{pmatrix}
        a & b\\
        c & d
    \end{pmatrix} \\
    \begin{pmatrix}
        b & -a+b\\
        d & -c+d
    \end{pmatrix}
    &=
    \begin{pmatrix}
        -c & -d\\
        a+c & b+d
    \end{pmatrix}
\end{align*}
Clearly, the elements which commute are:
\[
    \begin{pmatrix}
        -c+d & c\\
        -c & d
    \end{pmatrix}
    \wedge c^2 -c d + d^2 = 1 \wedge c,d \in \mathbb{Z}.
\]
This leaves the following commuting matrices.
\[
    \begin{pmatrix}
        0 & 1\\
        -1 & -1
    \end{pmatrix},
    \begin{pmatrix}
        1 & 1\\
        -1 & 0
    \end{pmatrix},
    \begin{pmatrix}
        -1 & 0\\
        0 & -1
    \end{pmatrix},
    \begin{pmatrix}
        1 & 0\\
        0 & 1
    \end{pmatrix},
    \begin{pmatrix}
        -1 & -1\\
        1 & 0
    \end{pmatrix},
    \begin{pmatrix}
        0 & -1\\
        1 & 1
    \end{pmatrix}.
\]


\subsection{Question 2.}
\noindent
Find the smallest $n>0$ such that $(ST)^n = I$, where
$S = \begin{pmatrix}
         0 & -1\\
          1 & 0 \end{pmatrix}$, and
$T = \begin{pmatrix}
         1 & 1\\
         0 & 1 \end{pmatrix}$.

\subsection*{Answer}
\noindent
The smallest $n>0$ is $6$, since $(ST)^6 = I$.


\subsection{Question 3 a).}
\noindent
Determine the point, in the fundamental region $R_\Gamma$, which is equivalent to $(8 + 6i)/(3 + 2i)$.

\subsection*{Answer}
\noindent
Since $(T^3\cdot S \cdot T^3)^{-1}(\frac{8+6i}{3+2i})=2i$, $\frac{-3(2i)-8}{-(2i)-3} = \frac{8+6i}{3+2i}$.


\subsection{Question 3 b).}
\noindent
Determine the point, in the fundamental region $R_\Gamma$, which is equivalent to $(11 + 10i)/(12 + 6i)$.

\subsection*{Answer}
\noindent
Since $(T \cdot S \cdot T^{-1})(\frac{11+10i}{12+6i})=\frac{5}{17}+\frac{54}{17}i$,
$\frac{-(\frac{5}{17}+\frac{54}{17}i)+2}{-(\frac{5}{17}+\frac{54}{17}i)-1} = \frac{11+10i}{12+6i}$.



\subsection{Question 4.}
\noindent
Determine all elements $A$ of $\Gamma$ which leave $i$ fixed.

\subsection*{Answer}
\noindent
\begin{align*}
    \frac{b + a i}{d + c i} &= i \\
    \frac{(b d + a c) + (a d - b c) i}{c^2 + d^2} &= i \\
    \left\{
    \begin{array}{ll}
        ac + bd &= 0\\
        c^2 + d^2 &= 1\\
        ad - bc &= 1
    \end{array}
    \right.
\end{align*}
Solving the last system of equations in the integers gives the following solutions for ${a,b,c,d}$: \\
$\{0,-1,1,0\},\{-1,0,0,-1\},\{0,1,-1,0\},\{1,0,0,1\}$ or in matrices: $S, S^2, S^3$ and $I$.


\subsection{Question 5.}
\noindent
Determine all elements $A$ of $\Gamma$ which leave $\rho = e^{\frac{2 \pi i}{3}} $ fixed.

\subsection*{Answer}
\noindent
\begin{align*}
    \frac{b + a (-\frac{1}{2}+\frac{1}{2}\sqrt{3}i)}{d + c (-\frac{1}{2}+\frac{1}{2}\sqrt{3}i)} &= (-\frac{1}{2}+\frac{1}{2}\sqrt{3}i) \\
    \frac{(a c - \frac{1}{2}b c - \frac{1}{2}a d + b d)+(- \frac{1}{2} bc + \frac{1}{2} ad)\sqrt{3}i}{c^2 - cd + d^2}&= (-\frac{1}{2}+\frac{1}{2}\sqrt{3}i) \\
    \left\{
    \begin{array}{ll}
        (a c - \frac{1}{2}b c - \frac{1}{2}a d + b d) &= -\frac{1}{2} \\
        (- \frac{1}{2} bc + \frac{1}{2} ad) &= \frac{1}{2}  \\
        c^2 - cd + d^2 &= 1 \\
        ad - bc &= 1
    \end{array}
    \right.
\end{align*}
Solving the last system of equations in the integers gives the following solutions for ${a,b,c,d}$: \\
$\{0,-1,1,1\},\{-1,-1,1,0\},\{-1,0,0,-1\},\{0,1,-1,-1\},\{1,1,-1,0\},\{1,0,0,1\}$ \\or in matrices:
$ST, (ST)^2, (ST)^3, (ST)^4, (ST)^5$ and $I$.


\subsection{Question 6.}
\noindent
If $x$ and $y$ are subjected to a unimodular transformation, say
\[
    x' = \alpha x + \beta y
\]
\[
    y' = \gamma x + \delta y
\]
where $\begin{pmatrix} \alpha & \beta \\ \gamma & \delta \end{pmatrix} \in \Gamma$, prove that $Q(x,y)$ gets
transformed to a quadratic form having the same discriminant.

\subsection*{Answer}
\noindent
\begin{align*}
    Q(x,y) &= (x, y) \cdot \begin{pmatrix} A & B/2 \\ B/2 & C \end{pmatrix} \cdot \begin{pmatrix} x \\ y \end{pmatrix} \\
    &= (x, y) \cdot {\begin{pmatrix} \alpha & \beta \\ \gamma & \delta \end{pmatrix}}^T \cdot \begin{pmatrix} A & B/2 \\ B/2 & C \end{pmatrix} \cdot \begin{pmatrix} \alpha & \beta \\ \gamma & \delta \end{pmatrix} \cdot \begin{pmatrix} x \\ y \end{pmatrix} \\
    &= (x, y) \cdot \begin{pmatrix} A \alpha^2 + B \alpha \gamma +C \gamma^2 & (2 A \alpha \beta + B \beta \gamma + B \alpha \delta +2 C \gamma \delta)/2 \\ (2 A \alpha \beta + B \beta \gamma + B \alpha \delta +2 C \gamma \delta)/2 & A \beta^2 + B \beta \delta +C \delta^2 \end{pmatrix} \cdot \begin{pmatrix} x \\ y \end{pmatrix}
\end{align*}
The discriminant of this transformed BQF is:
\begin{align*}
    d &= 4 (A \alpha^2 + B \alpha \gamma +C \gamma^2)(A \beta^2 + B \beta \delta +C \delta^2) - (2 A \alpha \beta + B \beta \gamma + B \alpha \delta +2 C \gamma \delta)^2 \\
    &= 4 A C (\alpha^2 \delta^2  - 2 \alpha \beta \gamma \delta + \beta^2 \delta^2) - B^2 (\alpha^2 \delta^2  - 2 \alpha \beta \gamma \delta + \beta^2 \gamma^2) \\
    &= ( 4 A C - B^2)(\alpha^2 \delta^2  - 2 \alpha \beta \gamma \delta + \beta^2 \gamma^2) \\
    &= ( 4 A C - B^2)(\alpha \delta - \beta \gamma)^2 \\
    &= 4 AC - B^2 .
\end{align*}


\subsection{Question 7 a).}
\noindent
Consider $Q(x,y)=a x^2 + b x y +c y^2$, with $a>0, c>0$ and $d>0$. The root $\tau$ of $a z^2 + b z + c = 0$ with
Im $\tau > 0$, is called the representative of $Q(x,y)$. Show that there is one-to-one correspondence between the
set of forms with discriminant d and the set of complex numbers $\tau$ with Im $\tau > 0$.

\subsection*{Answer}
\noindent
[TBD]

\subsection{Question 7 b).}
\noindent
Show that two BQF's with discriminant $d$ are equivalent if and only if their representatives are equivalent under $\Gamma$.
\subsection*{Answer}
\noindent
[TBD]

\subsection*{Answer}
\noindent
[TBD]


\subsection{Question 8.}
\noindent
Prove that a form $Q(x,y) = a x^2 + b x y +c y^2$ is reduced, if and only if, either $-a < b \leq a < c$ or
$0 \leq b \leq a = c$.

\subsection*{Answer}
\noindent
[TBD]


\subsection{Question 9.}
\noindent
Assume now that the form $Q(x, y) = a x^2 + b x y + c y^2$ has integer coefficients $a, b, c$.
Prove that for a given $d$ there are only a finite number of equivalence classes  with discriminant $d$.
This number is called the dass number and is denoted by $h(d)$.

\subsection*{Answer}
\noindent
[TBD]


\subsection{Question 10.}
\noindent
Determine all reduced forms with integer coefficients $a, b, c$ and the c1ass number $h(d)$ for each $d$ in the
interval $1 \leq d \leq 20$.

\subsection*{Answer}
\noindent
[TBD]


\subsection{Question 11.}
\noindent
Show that $\Gamma^{(n)} \leqslant \Gamma$. Moreover, if $B \in \Gamma^{(n)}$ then
$A^{-1} B A \in \Gamma^{(n)} \ \forall A \in \Gamma$. That is $\Gamma^{(n)} \mathrel{\unlhd} \Gamma$.

\subsection*{Answer}
\noindent
We use that if for any $A,B \in \Gamma^{(n)}$ we have $AB^{-1} \in \Gamma^{(n)}$ then $\Gamma^{(n)}$ is a subgroup
of $\Gamma$. Now, since $A,B,B^{-1} \mod n =I$ we have $AB^{-1} \mod n = I$ and thus $AB^{-1} \in \Gamma^{(n)}$.

\noindent
We use that if $B \in \Gamma, A \in \Gamma{(n)}$ we have $B^{-1}AB \in \Gamma^{(n)}$ then $\Gamma^{(n)}$ is a normal
subgroup of $\Gamma$. Now, since $A \mod n = I$, we have $B^{-1}A \mod n = B^{-1} \mod n$, and
thus $B^{-1}AB \mod n = I$ and thus $B^{-1}AB \in \Gamma^{(n)}$.


\subsection{Question 12.}
\noindent
The quotient group $\Gamma / \Gamma{(n)}$ is finite. That is, there exist a finite number of elements of
$\Gamma$, say $A_1 \cdots A_k$, such that every $B \in \Gamma$ is representable in the form
\[
    B=A_i B^{(n)}, \text{ where } 1 \leq i \leq k \text{ and } B^{(n)} \in \Gamma^{(n)}.
\]
The smallest such $k$ is called the index of $\Gamma^{(n)}$ in $\Gamma$.


\subsection*{Answer}
\noindent
[TBD]