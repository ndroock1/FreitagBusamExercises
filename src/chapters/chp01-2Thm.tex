\newpage
\section{Theorems}

\subsection*{Theorem 1.1}

\begin{proposition}[fundamental pair]
    $(\omega_1, \omega_2)$ is a fundamental pair IFF the triangle $(0, \omega_1, \omega_2)$ contains no further periods.
\end{proposition}

\begin{proof}
    . \\
    $\Rightarrow$ \\
    The set of points $z$ of the parallelogram $(0, \omega_1, \omega_1 + \omega_2, \omega_2)$ is defined by
    $
        \{z | z = \alpha \cdot \omega_1 + \beta \cdot \omega_2 \wedge 0 \leq \alpha \leq 1 \wedge 0 \leq \beta \leq 1 \}.
    $
    The only periods in this set are $(0, \omega_1, \omega_1 + \omega_2, \omega_2)$ so the triangle contains no periods
    other than the vertices. \\
    $\Leftarrow$ \\
    Let $\omega = t_1 \cdot \omega_1 + t_2 \cdot \omega_2$ $(t_1, t_2 \in \mathbf{R})$ be any period.
    If $t_1$ and $t_2$ are not integers we would have a period in the triangle,
    since $\omega - ( [t_1] \cdot \omega_1 + [t_2] \cdot \omega_2 ) = ( \{ t_1 \} \cdot \omega_1 + \{ t_2 \} \cdot \omega_2 )$.
    This is a contradiction, so $t_1$ and $t_2$ are integers.
\end{proof}


\subsection*{Theorem 1.3}

\begin{proposition}[]
    A non-constant elliptic function has a fundamental pair of periods.
\end{proposition}

\begin{proof}
    Let $\omega_1$ be a non-zero period with smallest modulus and smallest non-negative argument. If other periods
    exist, besides $-\omega_1$, with the same modulus, let this be $\omega_2$. If not, choose $\omega_2$ from the next
    larger circle containing periods $\neq n \cdot \omega_1$. We now have, by construction, no periods in the triangle
    $(0, \omega_1, \omega_2)$ other than the vertices. Hence, $(\omega_1, \omega_2)$ is fundamental.
\end{proof}


\subsection*{Theorem 1.4}

\begin{proposition}[]
    If an elliptic function has no poles in a period parallelogram, then the function is constant.
\end{proposition}

\begin{proof}
    If $f$ has no poles in a period parallelogram, then $f$ is continuous and hence bounded on the closure of the
    parallelogram. By periodicity, $f$ is bounded in the whole plane. Hence, by Liouville's Theorem, $f$ is constant.
\end{proof}


\subsection*{Theorem 1.5}

\begin{proposition}[]
    If an elliptic function has no zeros in a period parallelogram, then the function is constant.
\end{proposition}

\begin{proof}
    Apply Theorem 1.4 to the reciprocal $1/f$.
\end{proof}


\subsection*{Theorem 1.6}

\begin{proposition}[]
    The contour integral of an elliptic function taken along the boundary of any cell is zero.
\end{proposition}

\begin{proof}
    The integrals along parallel edges cancel because of periodicity.
\end{proof}


\subsection*{Theorem 1.7}

\begin{proposition}[]
    The sum of the residues of an elliptic function at its poles in any period parallelogram is zero.
\end{proposition}

\begin{proof}
    Apply Cauchy's residue theorem to a cell and use Theorem 1.6.
\end{proof}


\subsection*{Theorem 1.8}

\begin{proposition}[]
    The number of zeros of an elliptic function in any period parallelogram is equal to the number of poles,
    each counted with multiplicity.
\end{proposition}

\begin{proof}
    The integral
    \[
        \frac{1}{2 \pi i} \int_C \frac{f'(z)}{f(z)} dz
    \]
    taken around the boundary C of a cell, counts the difference between the number of zeros and the number of poles
    inside the cell. But $f'/f$ is elliptic with the same periods as $f$, and Theorem 1.6 tells us that this integral
    is zero.
\end{proof}