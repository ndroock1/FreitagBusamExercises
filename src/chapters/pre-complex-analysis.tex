\chapter{Complex Analysis}

\section{Numbers}

\subsection{Algebra}

\subsection{Conjugate}

\subsection{Polar Representation}

\subsection{Euler's Formula}

\subsection{De Moivre's Theorem}

\subsection{n-th Roots of Unity}

\subsection{Complex Exponents}

\subsection{Complex Logarithm}


\newpage
\section{Differentiation}

\subsection{Limits}

\subsection{Continuity}

\subsection{Derivatives}

\subsection{Rules for Differentiation}

\subsection{Cauchy-Riemann Equations}

\subsection{Analytic Functions}

\subsection{Harmonic Functions}


\newpage
\section{Elementary Complex Functions}

\subsection{Polynomials}

\subsection{Exponential}

\subsection{Trigonometric}

\subsection{Hyperbolic}

\subsection{Logarithmic}


\newpage
\section{Integration}

\subsection{Contours in the Complex Plane}

\subsection{Complex Line Integrals}

\subsection{Cauchy Integral Theorem}

\subsection{Cauchy Integral Formula}


\newpage
\section{Series and Residues}

\subsection{Sequences and Series}

\begin{definition}[Sequence]
\end{definition}

\begin{definition}[Series]
\end{definition}

\begin{theorem}[Null Test]
\end{theorem}

\begin{definition}[Geometric Series]
\end{definition}

\begin{definition}[Series $\frac{1}{n^p}$]
\end{definition}

\begin{theorem}[Sum/Product Rule]
\end{theorem}

\begin{theorem}[Comparison Test]
\end{theorem}

\begin{definition}[Absolute Convergence]
\end{definition}

\begin{theorem}[Absolute Convergence Test]
\end{theorem}

\begin{theorem}[Ratio Test]
\end{theorem}

\subsection{Power Series}

\subsection{Residues}
