\chapter{Complex Analysis}

\section{Numbers}

\subsection{Algebra}

\subsection{Conjugate}

\subsection{Polar Representation}

\subsection{Euler's Formula}

\subsection{De Moivre's Theorem}

\subsection{n-th Roots of Unity}

\subsection{Complex Exponents}

\subsection{Complex Logarithm}


\newpage
\section{Differentiation}

\subsection{Limits}

\subsection{Continuity}

\subsection{Derivatives}

\subsection{Rules for Differentiation}

\subsection{Cauchy-Riemann Equations}

\subsection{Analytic Functions}

\subsection{Harmonic Functions}


\newpage
\section{Elementary Complex Functions}

\subsection{Polynomials}

\subsection{Exponential}

\subsection{Trigonometric}

\subsection{Hyperbolic}

\subsection{Logarithmic}


\newpage
\section{Integration}

\subsection{Contours in the Complex Plane}

\subsection{Complex Line Integrals}

\subsection{Cauchy Integral Theorem}

\subsection{Cauchy Integral Formula}


\newpage
\section{Series and Residues}

\subsection{Sequences and Series}
%TODO : Determine subsection contents ( Update from Mma. )
\subsubsection{Convergent Series}
\subsubsection{Basic Series}
\subsubsection{Convergence Theorems}
\subsubsection{Absolute Convergence}


\subsection{Power Series}

\subsection{Taylor Series}

\subsection{Laurent Series}

\subsection{Zeros and Singularities}

\subsection{Residues}

\subsection{Residue Theorem}
