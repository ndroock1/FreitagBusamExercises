\section{Exercises}

\subsection*{Question 1.}
\noindent
Given two pairs of complex numbers $(\omega_1, \omega_2)$ and $({\omega_1}^\prime, {\omega_2}^\prime)$ with non-real
ratios $\omega_2 / \omega_1$ and ${\omega_2}^\prime / {\omega_1}^\prime$. Prove that they generate the same set of
periods iff there is a 2 by 2 matrix $\begin{pmatrix} a & b \\ c & d \end{pmatrix}$ with integer entries and
determinant $\pm 1$ such that \[
\begin{pmatrix} {\omega_2}^\prime \\ {\omega_1}^\prime \end{pmatrix} =
\begin{pmatrix} a & b \\ c & d \end{pmatrix} \begin{pmatrix} \omega_2 \\ \omega_1 \end{pmatrix}.
\]

\subsection*{Answer}
\noindent
With 'the set of periods' is meant here the lattice generated by two complex numbers.
We use vector notation for the complex numbers $(\omega_1, \omega_2)$ and $({\omega_1}^\prime, {\omega_2}^\prime)$.
The condition that $\omega_2 / \omega_1$ and ${\omega_2}^\prime / {\omega_1}^\prime$ have non-real ratios means,
in vector terminology, that $\omega_1, \omega_2$, resp. ${\omega_1}^\prime, {\omega_2}^\prime$ are both
non-zero and independent.
A 2 by 2 matrix $\begin{pmatrix} a & b \\ c & d \end{pmatrix}$ with integer entries and determinant $\pm 1$ is called
a unimodular matrix. The inverse of a unimodular matrix is also unimodular.
The question we address here is how to determine if two given bases $\omega = (\omega_1, \omega_2)$ and
$\omega^\prime = ({\omega_1}^\prime, {\omega_2}^\prime)$ are equivalent, i.e. generate the same lattice
$L(\omega)=L(\omega^\prime)$.
We answer this by proving the following proposition.

\begin{proposition}
    The lattices $L(\omega)$ and $L(\omega^\prime)$ are equivalent if and only if $\omega^\prime = \omega \cdot U.$
\end{proposition}

\begin{proof}
    Assume $L(\omega)$ = $L(\omega^\prime)$, then unimodular matrices exist, such that:
    $\omega^\prime = \omega \cdot U$, similarly $\omega = \omega^\prime \cdot V$.
\end{proof}