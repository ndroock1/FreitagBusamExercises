\section{Exercises}

\subsection{Question 1.}
\noindent
Given two pairs of complex numbers $(\omega_1, \omega_2)$ and $({\omega_1}^\prime, {\omega_2}^\prime)$ with non-real
ratios $\omega_2 / \omega_1$ and ${\omega_2}^\prime / {\omega_1}^\prime$. Prove that they generate the same set of
periods iff there is a 2 by 2 matrix $\begin{pmatrix} a & b \\ c & d \end{pmatrix}$ with integer entries and
determinant $\pm 1$ such that \[
\begin{pmatrix} {\omega_2}^\prime \\ {\omega_1}^\prime \end{pmatrix} =
\begin{pmatrix} a & b \\ c & d \end{pmatrix} \begin{pmatrix} \omega_2 \\ \omega_1 \end{pmatrix}.
\]

\subsection*{Answer}
\noindent
With 'the set of periods' is meant here the lattice generated by two complex numbers.
We use vector notation for the complex numbers $(\omega_1, \omega_2)$ and $({\omega_1}^\prime, {\omega_2}^\prime)$.
The condition that $\omega_2 / \omega_1$ and ${\omega_2}^\prime / {\omega_1}^\prime$ have non-real ratios means,
in vector terminology, that $\omega_1, \omega_2$, resp. ${\omega_1}^\prime, {\omega_2}^\prime$ are both
non-zero and independent.
A 2 by 2 matrix $\begin{pmatrix} a & b \\ c & d \end{pmatrix}$ with integer entries and determinant $\pm 1$ is called
a unimodular matrix. The inverse of a unimodular matrix is also unimodular by Cramer's Rule for the inverse.
The question we address here is how to determine if two given bases $\omega = (\omega_1, \omega_2)$ and
$\omega^\prime = ({\omega_1}^\prime, {\omega_2}^\prime)$ are equivalent, i.e. generate the same lattice
$L(\omega)=L(\omega^\prime)$.
We answer this by proving the following proposition.

\begin{proposition}
    The lattices $L(\omega)$ and $L(\omega^\prime)$ are equivalent if and only if $\omega^\prime = \omega \cdot U.$
\end{proposition}

\begin{proof}
    Assume $L(\omega)$ = $L(\omega^\prime)$, then integer matrices exist, such that:
    $\omega^\prime = \omega \cdot U$, and similarly $\omega = \omega^\prime \cdot V$. Hence
    \begin{align*}
        \omega^\prime &= \omega^\prime \cdot V \cdot U \\
        {\omega^\prime}^T \cdot \omega^\prime &= {(\omega^\prime \cdot V \cdot U)}^T \cdot (\omega^\prime \cdot V \cdot U) &&\text{transpose both sides} \\
        {\omega^\prime}^T \cdot \omega^\prime &= {(V \cdot U)}^T \cdot ({\omega^\prime}^T \cdot \omega^\prime) \cdot (V \cdot U)\\
        \det{({\omega^\prime}^T \cdot \omega^\prime)} &= \det{({(V \cdot U)}^T \cdot ({\omega^\prime}^T \cdot \omega^\prime) \cdot (V \cdot U))} &&\text{taking determinants} \\
        \det{({\omega^\prime}^T \cdot \omega^\prime)} &= \det{({(V \cdot U)})}^2 \cdot \det{({\omega^\prime}^T \cdot \omega^\prime)} \\
        \det{(V)}\det{(U)}  &= \pm 1
    \end{align*}

    Since both $U,V$ are integer matrices, we conclude that $\det{(U)} = \pm 1$ and that $U$ is unimodular.\\

    For the other direction, assume that $\omega^\prime = \omega \cdot U$ for some unimodular matrix $U$.
    Therefore each column of $\omega^\prime$ is contained in $L(\omega)$ and we get $L(\omega^\prime) \subseteq L(\omega)$.
    In addition, $L(\omega) = \omega^\prime \cdot U^{-1}$, and since $U^{-1}$ is unimodular we similarly get that
    $L(\omega) \subseteq L(\omega^\prime)$. We conclude that $L(\omega)=L(\omega^\prime)$.
\end{proof}


\subsection{Question 2.}
\noindent
Let $S(0)$ denote the sum of the zeros of an elliptic function $f$ in a period parallelogram. Prove that
$S(0)-S(\infty)$ is a period of $f$. (Hint: Integrate $\frac{z \cdot f'(z)}{f(z)}$).

\subsection*{Answer}
\noindent
So we need to prove that inside a fundamental parallelogram spanned by the complex numbers $(\omega_1, \omega_2)$ the
sum of the coordinates of the zeroes equals the sum of the coordinates of the poles modulo a period of $f$. We have to
prove the following proposition.

\begin{proposition}
    \[
        S(0) - S(\infty) = m \cdot \omega_1 + n \cdot \omega_2  \  ( \text{ for some } m,n \in \mathbb{Z} \  )
    \]
\end{proposition}

\begin{proof}
    \begin{align*}
        S(0) - S(\infty) &= \frac{1}{2 \pi i} \oint_{C} z \frac{f'(z)}{f(z)} dz \\
        &= \frac{1}{2 \pi i} \int_{0}^{\omega_1} z \frac{f'(z)}{f(z)} dz +
                            \frac{1}{2 \pi i} \int_{\omega_1}^{\omega_1 + \omega_2} z \frac{f'(z)}{f(z)} dz +
                            \frac{1}{2 \pi i} \int_{\omega_1 + \omega_2}^{\omega_2} z \frac{f'(z)}{f(z)} dz +
                            \frac{1}{2 \pi i} \int_{\omega_2}^{0} z \frac{f'(z)}{f(z)} dz \\
        &= \frac{1}{2 \pi i} ( \int_{0}^{\omega_1} z \frac{f'(z)}{f(z)} dz +
         \int_{\omega_1 + \omega_2}^{\omega_2} z \frac{f'(z)}{f(z)} dz +
         \int_{\omega_1}^{\omega_1 + \omega_2} z \frac{f'(z)}{f(z)} dz +
         \int_{\omega_2}^{0} z \frac{f'(z)}{f(z)} dz ) \\
        &= \frac{1}{2 \pi i} ( \int_{0}^{\omega_1} z \frac{f'(z)}{f(z)} dz -
        \int_{\omega_2}^{\omega_1 + \omega_2} z \frac{f'(z)}{f(z)} dz +
        \int_{\omega_1}^{\omega_1 + \omega_2} z \frac{f'(z)}{f(z)} dz -
        \int_{0}^{\omega_2} z \frac{f'(z)}{f(z)} dz ) \\
        &= \frac{1}{2 \pi i} ( \int_{0}^{\omega_1} z \frac{f'(z)}{f(z)} dz -
        \int_{0}^{\omega_1} (z + \omega_2) \frac{f'(z+\omega_2)}{f(z+\omega_2)} dz +
        \int_{0}^{\omega_2} (z + \omega_1) \frac{f'(z+\omega_1)}{f(z+\omega_1)} dz -
        \int_{0}^{\omega_2} z \frac{f'(z)}{f(z)} dz ) \\
        &= \frac{1}{2 \pi i} ( \int_{0}^{\omega_1} z \frac{f'(z)}{f(z)} dz -
        \int_{0}^{\omega_1} (z + \omega_2) \frac{f'(z)}{f(z)} dz +
        \int_{0}^{\omega_2} (z + \omega_1) \frac{f'(z)}{f(z)} dz -
        \int_{0}^{\omega_2} z \frac{f'(z)}{f(z)} dz ) \\
        &= \frac{1}{2 \pi i} ( \int_{0}^{\omega_1} ( z - (z + \omega_2) ) \frac{f'(z)}{f(z)} dz +
        \int_{0}^{\omega_2} ((z + \omega_1) -z ) \frac{f'(z)}{f(z)} dz ) \\
        &= \frac{1}{2 \pi i} ( \omega_2 \int_{0}^{\omega_1} \frac{f'(z)}{f(z)} dz +
        \omega_1 \int_{0}^{\omega_2} \frac{f'(z)}{f(z)} dz ) \\
        &= \frac{1}{2 \pi i} ( \omega_2 \log{f(z)} \rvert_{0}^{\omega_1} + \omega_1 \log{f(z)} \rvert_{0}^{\omega_2} ) \\
        &= \frac{1}{2 \pi i} ( \omega_2 \log{1} + \omega_1 \log{1} ) \\
        &= \frac{1}{2 \pi i} ( \omega_2 \cdot n \cdot 2 \pi i + \omega_1 \cdot m \cdot 2 \pi i ) \\
        & = m \cdot \omega_1 + n \cdot \omega_2
    \end{align*}
\end{proof}


\subsection{Question 3 a).}
\noindent
Prove that $\wp(u)=\wp(v)$, if and only if, $u-v$ or $u+v$ is a period of $\wp$.

\subsection*{Answer}
\noindent
So we have to prove the following proposition.

\begin{proposition}
    \[
        \wp(u)=\wp(v) \text{ if and only if } u-v \text{ or } u+v \in m \cdot \omega_1 + n \cdot \omega_2
        \  ( \text{ for some } m,n \in \mathbb{Z} \  )
    \]
\end{proposition}

\begin{proof}
    Assume $\wp(u) = \wp(v)$, then by \textbf{periodicity} of $\wp$ : $u= \pm v+\omega \implies u \pm v \in \omega$.
    In the other direction assume $u \pm v \in \omega$, then $u= \pm v+\omega$ and by \textbf{periodicity}
    $\wp(u) = \wp(v)$.
\end{proof}


\subsection{Question 3 b).}
\noindent
Let $a_1, \cdots , a_n$ and $b_1, \cdots , b_m$ be complex numbers such that none of the numbers $\wp(a_i) - \wp(b_j)$
is zero. Let
\[
    f(z) = \prod_{i=1}^{n} [ \wp(z) - \wp(a_i) ] / \prod_{r=1}^{m} [ \wp(z) - \wp(b_r) ].
\]
Prove that $f$ is an even elliptic function with zeros at $a_1, \cdots , a_n$ and poles at $b_1, \cdots , b_m$.

\subsection*{Answer}
\noindent
We have to prove that $f$ is elliptic, that $f$ is even, that $f$ has zeros at $a_1, \cdots , a_n$ and that $f$
has poles at $b_1, \cdots , b_m$. We also need to explain the impact of any of the $\wp(a_i) - \wp(b_j)$ being zero.

\begin{proof}
    The sum, difference, product and quotient of elliptic functions are also elliptic functions are also elliptic
    functions, since the set of all elliptic functions for a fixed lattice is a field. Hence, $f(z)$ is elliptic.\\
    The function $f(z)$ is even because
    \[
        f(z) = \prod_{i=1}^{n} [ \wp(z) - \wp(a_i) ] / \prod_{r=1}^{m} [ \wp(z) - \wp(b_r) ] =
        \prod_{i=1}^{n} [ \wp(-z) - \wp(a_i) ] / \prod_{r=1}^{m} [ \wp(-z) - \wp(b_r) ] =
        f(-z).
    \]
    The function $f(z)$ has zeros at $a_1, \cdots , a_n$ since $\wp(a_i) - \wp(a_i) = 0$ and has poles at
    $b_1, \cdots , b_m$ since $\wp(b_j) - \wp(b_j) = 0$.\\
    If $\wp(a_i) - \wp(b_j) = 0$ for some $i,j$ then $f(z)$ might be undefined.
\end{proof}


\subsection{Question 4.}
\noindent
Prove that every even elliptic function $f$ is a rational function of $\wp$, where the periods of $\wp$ are a subset of
the periods of $f$.

\subsection*{Answer}
\noindent
We answer this by proving the following two propositions.

\begin{proposition}[1]
    Let $f$ be an even elliptic function, whose pole set is contained in $L$. Then $f$can be represented as a
    polynomial in $\wp$. The degree of this polynomial is half of the order of $f$.
\end{proposition}

\begin{proof}
    The Laurent series of $f$ in $0$ has only even coefficients, and is hence of the form
    \[
        f(z) = a_{-2n}z^{-2n}+a_{-2(n-1)}z^{-2(n-1)}+ \cdots \ \ \ \ n \geq 1, a_{-2n} \neq 0.
    \]
    The Laurent series of $\wp$ is of the form
    \[
        \wp(z) = z^{-2} + \cdots \ ,
    \]
    and for the n-th power we obtain
    \[
        \wp(z)^n = z^{-2n} + \cdots
    \]
    Just as $f$ and $\wp$ the function
    \[
        g(z) = f(z) - a_{-2n} \wp(z)^n
    \]
    is an even elliptic function whose pole set is contained in $L$. The order of $g$ is strictly smaller than the
    order of $f$. The proof now is obtained by induction.
\end{proof}


\begin{proposition}[2]
    The field of all even elliptic functions for the lattice $L$ is equal to $\mathbb{C}(\wp(z))$, as a subfield
    of $K(L)$, and is thus isomorphic to the field of rational functions.
\end{proposition}

\begin{proof}
    Let $f$ be a non-constant even elliptic function. If $a$ is a pole of $f$ which is not contained in $L$, the
    function
    \[
        z \mapsto (\wp(z)-\wp(a))^N \cdot f(z)
    \]
    has in $z=a$ a removable singularity, if $N$ is sufficiently large. Since $f$ has only finitely many poles mod $L$,
    we find finitely many points $a_1, \cdots , a_m$ and natural numbers $N_1, \cdots , N_m$ such that
    \[
        g(z) = f(z) \cdot \prod_{j=1}^{n} [ \wp(z) - \wp(a_j) ]^{N_j}
    \]
    has no poles outside $L$. By proposition 1 $g(z)$ is a polynomial in $\wp(z)$.
\end{proof}


\subsection{Question 5.}
\noindent
Prove that every elliptic function $f$ can be expressed in the form
\[
    f(z) = R_1(\wp(z))+ \wp'(z)R_2(\wp(z)),
\]
where $R_1, R_2$ are rational functions and $\wp$ has the same set of periods as $f$.

\subsection*{Answer}

\begin{proof}
    Let $f$ be an elliptic function, $E_i$ an even elliptic function and $O_i$ an odd elliptic function, all for the
    lattice $L$, then

    \begin{align*}
        f(z) &= \frac{f(z)+f(-z)}{2}+\frac{f(z)-f(-z)}{2} \\
        f(z) &= E_1(z)+O_1(z) \\
        f(z) &= E_1(z)+\wp'(z) \frac{O_1(z)}{\wp'(z)} \\
        f(z) &= E_1(z)+\wp'(z) E_2(z) \Rightarrow \\
        f(z) &= R_1(\wp(z))+ \wp'(z)R_2(\wp(z))
    \end{align*}
\end{proof}


\subsection{Question 6.}
\noindent
Let $f$ and $g$ be two elliptic functions with the same set of periods. Prove that there exists a polynomial $P(x, y)$,
not identically zero, such that
\[
    P(f(z), g(z) )= C
\]
where $C$ is a constant (depending on $f$ and $g$ but not on $z$).

\subsection*{Answer}
\noindent
We answer this by proving the following proposition.

\begin{proposition}
    $1=1$
\end{proposition}

\begin{proof}
    Assume that:

    \begin{align*}
        1 &= 1
    \end{align*}

\end{proof}


\subsection{Question 9.}
\noindent
According to Exercise 4, the function $\wp(2z)$ is a rational function of $\wp(z)$. Prove that, in fact,
\[
    \wp(2z) = \frac{(\wp(z)^2 + \frac{1}{4}g_2)^2+2g_3\wp(z)}{4\wp(z)^3-g_2\wp(z)-g_3}.
\]

\subsection*{Answer}
\noindent
We answer this by proving the following proposition.

\begin{proposition}
    $1=1$
\end{proposition}

\begin{proof}
    Assume that:

    \begin{align*}
        1 &= 1
    \end{align*}

\end{proof}
