\section{Exercises}

\subsection*{Question 1.}
\noindent
Given two pairs of complex numbers $(\omega_1, \omega_2)$ and $({\omega_1}^\prime, {\omega_2}^\prime)$ with non-real
ratios $\omega_2 / \omega_1$ and ${\omega_2}^\prime / {\omega_1}^\prime$. Prove that they generate the same set of
periods iff there is a 2 by 2 matrix $\begin{pmatrix} a & b \\ c & d \end{pmatrix}$ with integer entries and
determinant $\pm 1$ such that \[
\begin{pmatrix} {\omega_2}^\prime \\ {\omega_1}^\prime \end{pmatrix} =
\begin{pmatrix} a & b \\ c & d \end{pmatrix} \begin{pmatrix} \omega_2 \\ \omega_1 \end{pmatrix}.
\]

\subsection*{Answer}
\noindent
With 'the set of periods' is meant here the lattice generated by two complex numbers.
We use vector notation for the complex numbers $(\omega_1, \omega_2)$ and $({\omega_1}^\prime, {\omega_2}^\prime)$.
The condition that $\omega_2 / \omega_1$ and ${\omega_2}^\prime / {\omega_1}^\prime$ have non-real ratios means,
in vector terminology, that $\omega_1, \omega_2$, resp. ${\omega_1}^\prime, {\omega_2}^\prime$ are both
non-zero and independent.
A 2 by 2 matrix $\begin{pmatrix} a & b \\ c & d \end{pmatrix}$ with integer entries and determinant $\pm 1$ is called
a unimodular matrix. The inverse of a unimodular matrix is also unimodular.
The question we address here is how to determine if two given bases $\omega = (\omega_1, \omega_2)$ and
$\omega^\prime = ({\omega_1}^\prime, {\omega_2}^\prime)$ are equivalent, i.e. generate the same lattice
$L(\omega)=L(\omega^\prime)$.
We answer this by proving the following proposition.

\begin{proposition}
    The lattices $L(\omega)$ and $L(\omega^\prime)$ are equivalent if and only if $\omega^\prime = \omega \cdot U.$
\end{proposition}

\begin{proof}
    Assume $L(\omega)$ = $L(\omega^\prime)$, then integer matrices exist, such that:
    $\omega^\prime = \omega \cdot U$, and similarly $\omega = \omega^\prime \cdot V$. Hence
    \begin{align*}
        \omega^\prime &= \omega^\prime \cdot V \cdot U \\
        {\omega^\prime}^T \cdot \omega^\prime &= {(\omega^\prime \cdot V \cdot U)}^T \cdot (\omega^\prime \cdot V \cdot U) &&\text{transpose both sides} \\
        {\omega^\prime}^T \cdot \omega^\prime &= {(V \cdot U)}^T \cdot ({\omega^\prime}^T \cdot \omega^\prime) \cdot (V \cdot U)\\
        \det{({\omega^\prime}^T \cdot \omega^\prime)} &= \det{({(V \cdot U)}^T \cdot ({\omega^\prime}^T \cdot \omega^\prime) \cdot (V \cdot U))} &&\text{taking determinants} \\
        \det{({\omega^\prime}^T \cdot \omega^\prime)} &= \det{({(V \cdot U)})}^2 \cdot \det{({\omega^\prime}^T \cdot \omega^\prime)} \\
        \det{(V)}\det{(U)}  &= \pm 1
    \end{align*}

    Since both $U,V$ are integer matrices, we conclude that $\det{(U)} = \pm 1$ and that $U$ is unimodular.\\

    For the other direction, assume that $\omega^\prime = \omega \cdot U$ for some unimodular matrix $U$.
    Therefore each column of $\omega^\prime$ is contained in $L(\omega)$ and we get $L(\omega^\prime) \subseteq L(\omega)$.
    In addition, $L(\omega) = \omega^\prime \cdot U^{-1}$, and since $U^{-1}$ is unimodular we similarly get that
    $L(\omega) \subseteq L(\omega^\prime)$. We conclude that $L(\omega)=L(\omega^\prime)$.
\end{proof}


\subsection*{Question 2.}
\noindent
Let $S(0)$ denote the sum of the zeros of an elliptic function $f$ in a period parallelogram. Prove that
$S(0)-S(\infty)$ is a period of $f$. (Hint: Integrate $\frac{z \cdot f'(z)}{f(z)}$).

\subsection*{Answer}
\noindent
So we need to prove that inside a fundamental parallelogram spanned by the complex numbers $(\omega_1, \omega_2)$ the
sum of the coordinates of the zeroes equals the sum of the coordinates of the poles modulo a period of $f$. We have to
prove the following proposition.

\begin{proposition}
    \[
        S(0) - S(\infty) = m \cdot \omega_1 + n \cdot \omega_2  \  ( \text{ for some } m,n \in \mathbb{Z} \  )
    \]
\end{proposition}

\begin{proof}
    \begin{align*}
        S(0) - S(\infty) &= \frac{1}{2 \pi i} \oint_{C} z \frac{f'(z)}{f(z)} dz \\
        &= \frac{1}{2 \pi i} \int_{0}^{\omega_1} z \frac{f'(z)}{f(z)} dz +
                            \frac{1}{2 \pi i} \int_{\omega_1}^{\omega_1 + \omega_2} z \frac{f'(z)}{f(z)} dz +
                            \frac{1}{2 \pi i} \int_{\omega_1 + \omega_2}^{\omega_2} z \frac{f'(z)}{f(z)} dz +
                            \frac{1}{2 \pi i} \int_{\omega_2}^{0} z \frac{f'(z)}{f(z)} dz \\
        &= \frac{1}{2 \pi i} ( \int_{0}^{\omega_1} z \frac{f'(z)}{f(z)} dz +
         \int_{\omega_1 + \omega_2}^{\omega_2} z \frac{f'(z)}{f(z)} dz +
         \int_{\omega_1}^{\omega_1 + \omega_2} z \frac{f'(z)}{f(z)} dz +
         \int_{\omega_2}^{0} z \frac{f'(z)}{f(z)} dz ) \\
        &= \frac{1}{2 \pi i} ( \int_{0}^{\omega_1} z \frac{f'(z)}{f(z)} dz -
        \int_{\omega_2}^{\omega_1 + \omega_2} z \frac{f'(z)}{f(z)} dz +
        \int_{\omega_1}^{\omega_1 + \omega_2} z \frac{f'(z)}{f(z)} dz -
        \int_{0}^{\omega_2} z \frac{f'(z)}{f(z)} dz ) \\
        &= \frac{1}{2 \pi i} ( \int_{0}^{\omega_1} z \frac{f'(z)}{f(z)} dz -
        \int_{0}^{\omega_1} (z + \omega_2) \frac{f'(z+\omega_2)}{f(z+\omega_2)} dz +
        \int_{0}^{\omega_2} (z + \omega_1) \frac{f'(z+\omega_1)}{f(z+\omega_1)} dz -
        \int_{0}^{\omega_2} z \frac{f'(z)}{f(z)} dz ) \\
        &= \frac{1}{2 \pi i} ( \int_{0}^{\omega_1} z \frac{f'(z)}{f(z)} dz -
        \int_{0}^{\omega_1} (z + \omega_2) \frac{f'(z)}{f(z)} dz +
        \int_{0}^{\omega_2} (z + \omega_1) \frac{f'(z)}{f(z)} dz -
        \int_{0}^{\omega_2} z \frac{f'(z)}{f(z)} dz ) \\
        &= \frac{1}{2 \pi i} ( \int_{0}^{\omega_1} ( z - (z + \omega_2) ) \frac{f'(z)}{f(z)} dz +
        \int_{0}^{\omega_2} ((z + \omega_1) -z ) \frac{f'(z)}{f(z)} dz ) \\
        &= \frac{1}{2 \pi i} ( \omega_2 \int_{0}^{\omega_1} \frac{f'(z)}{f(z)} dz +
        \omega_1 \int_{0}^{\omega_2} \frac{f'(z)}{f(z)} dz ) \\
        &= \frac{1}{2 \pi i} ( \omega_2 \log{f(z)} \rvert_{0}^{\omega_1} + \omega_1 \log{f(z)} \rvert_{0}^{\omega_2} ) \\
        &= \frac{1}{2 \pi i} ( \omega_2 \log{1} + \omega_1 \log{1} ) \\
        &= \frac{1}{2 \pi i} ( \omega_2 \cdot n \cdot 2 \pi i + \omega_1 \cdot m \cdot 2 \pi i ) \\
        & = m \cdot \omega_1 + n \cdot \omega_2
    \end{align*}
\end{proof}


\subsection*{Question 3 a).}
\noindent
Prove that $\wp(u)=\wp(v)$, if and only if, $u-v$ or $u+v$ is a period of $\wp$.

\subsection*{Answer}
\noindent
So we have to prove the following proposition.

\begin{proposition}
    \[
        \wp(u)=\wp(v) \text{ if and only if } u \pm v \in m \cdot \omega_1 + n \cdot \omega_2
        \  ( \text{ for some } m,n \in \mathbb{Z} \  )
    \]
\end{proposition}

\begin{proof}
    \begin{align*}
        1 &= 1
    \end{align*}
\end{proof}