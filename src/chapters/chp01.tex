\section{Exercises}

\subsection{Question 1.}
\noindent
Given two pairs of complex numbers $(\omega_1, \omega_2)$ and $({\omega_1}^\prime, {\omega_2}^\prime)$ with non-real
ratios $\omega_2 / \omega_1$ and ${\omega_2}^\prime / {\omega_1}^\prime$. Prove that they generate the same set of
periods iff there is a 2 by 2 matrix $\begin{pmatrix} a & b \\ c & d \end{pmatrix}$ with integer entries and
determinant $\pm 1$ such that \[
\begin{pmatrix} {\omega_2}^\prime \\ {\omega_1}^\prime \end{pmatrix} =
\begin{pmatrix} a & b \\ c & d \end{pmatrix} \begin{pmatrix} \omega_2 \\ \omega_1 \end{pmatrix}.
\]

\subsection*{Answer}
\noindent
With 'the set of periods' is meant here the lattice generated by two complex numbers.
We use vector notation for the complex numbers $(\omega_1, \omega_2)$ and $({\omega_1}^\prime, {\omega_2}^\prime)$.
The condition that $\omega_2 / \omega_1$ and ${\omega_2}^\prime / {\omega_1}^\prime$ have non-real ratios means,
in vector terminology, that $\omega_1, \omega_2$, resp. ${\omega_1}^\prime, {\omega_2}^\prime$ are both
non-zero and independent.
A 2 by 2 matrix $\begin{pmatrix} a & b \\ c & d \end{pmatrix}$ with integer entries and determinant $\pm 1$ is called
a unimodular matrix. The inverse of a unimodular matrix is also unimodular by Cramer's Rule for the inverse.
The question we address here is how to determine if two given bases $\omega = (\omega_1, \omega_2)$ and
$\omega^\prime = ({\omega_1}^\prime, {\omega_2}^\prime)$ are equivalent, i.e. generate the same lattice
$L(\omega)=L(\omega^\prime)$.
We answer this by proving the following proposition.

\begin{proposition}
    The lattices $L(\omega)$ and $L(\omega^\prime)$ are equivalent if and only if $\omega^\prime = \omega \cdot U.$
\end{proposition}

\begin{proof}
    Assume $L(\omega)$ = $L(\omega^\prime)$, then integer matrices exist, such that:
    $\omega^\prime = \omega \cdot U$, and similarly $\omega = \omega^\prime \cdot V$. Hence
    \begin{align*}
        \omega^\prime &= \omega^\prime \cdot V \cdot U \\
        {\omega^\prime}^T \cdot \omega^\prime &= {(\omega^\prime \cdot V \cdot U)}^T \cdot (\omega^\prime \cdot V \cdot U) &&\text{transpose both sides} \\
        {\omega^\prime}^T \cdot \omega^\prime &= {(V \cdot U)}^T \cdot ({\omega^\prime}^T \cdot \omega^\prime) \cdot (V \cdot U)\\
        \det{({\omega^\prime}^T \cdot \omega^\prime)} &= \det{({(V \cdot U)}^T \cdot ({\omega^\prime}^T \cdot \omega^\prime) \cdot (V \cdot U))} &&\text{taking determinants} \\
        \det{({\omega^\prime}^T \cdot \omega^\prime)} &= \det{({(V \cdot U)})}^2 \cdot \det{({\omega^\prime}^T \cdot \omega^\prime)} \\
        \det{(V)}\det{(U)}  &= \pm 1
    \end{align*}

    Since both $U,V$ are integer matrices, we conclude that $\det{(U)} = \pm 1$ and that $U$ is unimodular.\\

    For the other direction, assume that $\omega^\prime = \omega \cdot U$ for some unimodular matrix $U$.
    Therefore each column of $\omega^\prime$ is contained in $L(\omega)$ and we get $L(\omega^\prime) \subseteq L(\omega)$.
    In addition, $L(\omega) = \omega^\prime \cdot U^{-1}$, and since $U^{-1}$ is unimodular we similarly get that
    $L(\omega) \subseteq L(\omega^\prime)$. We conclude that $L(\omega)=L(\omega^\prime)$.
\end{proof}


\subsection{Question 2.}
\noindent
Let $S(0)$ denote the sum of the zeros of an elliptic function $f$ in a period parallelogram. Prove that
$S(0)-S(\infty)$ is a period of $f$. (Hint: Integrate $\frac{z \cdot f'(z)}{f(z)}$).

\subsection*{Answer}
\noindent
So we need to prove that inside a fundamental parallelogram spanned by the complex numbers $(\omega_1, \omega_2)$ the
sum of the coordinates of the zeroes equals the sum of the coordinates of the poles modulo a period of $f$. We have to
prove the following proposition.

\begin{proposition}
    \[
        S(0) - S(\infty) = m \cdot \omega_1 + n \cdot \omega_2  \  ( \text{ for some } m,n \in \mathbb{Z} \  )
    \]
\end{proposition}

\begin{proof}
    \begin{align*}
        S(0) - S(\infty) &= \frac{1}{2 \pi i} \oint_{C} z \frac{f'(z)}{f(z)} dz \\
        &= \frac{1}{2 \pi i} \int_{0}^{\omega_1} z \frac{f'(z)}{f(z)} dz +
                            \frac{1}{2 \pi i} \int_{\omega_1}^{\omega_1 + \omega_2} z \frac{f'(z)}{f(z)} dz +
                            \frac{1}{2 \pi i} \int_{\omega_1 + \omega_2}^{\omega_2} z \frac{f'(z)}{f(z)} dz +
                            \frac{1}{2 \pi i} \int_{\omega_2}^{0} z \frac{f'(z)}{f(z)} dz \\
        &= \frac{1}{2 \pi i} ( \int_{0}^{\omega_1} z \frac{f'(z)}{f(z)} dz +
         \int_{\omega_1 + \omega_2}^{\omega_2} z \frac{f'(z)}{f(z)} dz +
         \int_{\omega_1}^{\omega_1 + \omega_2} z \frac{f'(z)}{f(z)} dz +
         \int_{\omega_2}^{0} z \frac{f'(z)}{f(z)} dz ) \\
        &= \frac{1}{2 \pi i} ( \int_{0}^{\omega_1} z \frac{f'(z)}{f(z)} dz -
        \int_{\omega_2}^{\omega_1 + \omega_2} z \frac{f'(z)}{f(z)} dz +
        \int_{\omega_1}^{\omega_1 + \omega_2} z \frac{f'(z)}{f(z)} dz -
        \int_{0}^{\omega_2} z \frac{f'(z)}{f(z)} dz ) \\
        &= \frac{1}{2 \pi i} ( \int_{0}^{\omega_1} z \frac{f'(z)}{f(z)} dz -
        \int_{0}^{\omega_1} (z + \omega_2) \frac{f'(z+\omega_2)}{f(z+\omega_2)} dz +
        \int_{0}^{\omega_2} (z + \omega_1) \frac{f'(z+\omega_1)}{f(z+\omega_1)} dz -
        \int_{0}^{\omega_2} z \frac{f'(z)}{f(z)} dz ) \\
        &= \frac{1}{2 \pi i} ( \int_{0}^{\omega_1} z \frac{f'(z)}{f(z)} dz -
        \int_{0}^{\omega_1} (z + \omega_2) \frac{f'(z)}{f(z)} dz +
        \int_{0}^{\omega_2} (z + \omega_1) \frac{f'(z)}{f(z)} dz -
        \int_{0}^{\omega_2} z \frac{f'(z)}{f(z)} dz ) \\
        &= \frac{1}{2 \pi i} ( \int_{0}^{\omega_1} ( z - (z + \omega_2) ) \frac{f'(z)}{f(z)} dz +
        \int_{0}^{\omega_2} ((z + \omega_1) -z ) \frac{f'(z)}{f(z)} dz ) \\
        &= \frac{1}{2 \pi i} ( \omega_2 \int_{0}^{\omega_1} \frac{f'(z)}{f(z)} dz +
        \omega_1 \int_{0}^{\omega_2} \frac{f'(z)}{f(z)} dz ) \\
        &= \frac{1}{2 \pi i} ( \omega_2 \log{f(z)} \rvert_{0}^{\omega_1} + \omega_1 \log{f(z)} \rvert_{0}^{\omega_2} ) \\
        &= \frac{1}{2 \pi i} ( \omega_2 \log{1} + \omega_1 \log{1} ) \\
        &= \frac{1}{2 \pi i} ( \omega_2 \cdot n \cdot 2 \pi i + \omega_1 \cdot m \cdot 2 \pi i ) \\
        & = m \cdot \omega_1 + n \cdot \omega_2
    \end{align*}
\end{proof}


\subsection{Question 3 a).}
\noindent
Prove that $\wp(u)=\wp(v)$, if and only if, $u-v$ or $u+v$ is a period of $\wp$.

\subsection*{Answer}
\noindent
So we have to prove the following proposition.

\begin{proposition}
    \[
        \wp(u)=\wp(v) \text{ if and only if } u-v \text{ or } u+v \in m \cdot \omega_1 + n \cdot \omega_2
        \  ( \text{ for some } m,n \in \mathbb{Z} \  )
    \]
\end{proposition}

\begin{proof}
    Assume $\wp(u) = \wp(v)$, then by \textbf{periodicity} of $\wp$ : $u= \pm v+\omega \implies u \pm v \in \omega$.
    In the other direction assume $u \pm v \in \omega$, then $u= \pm v+\omega$ and by \textbf{periodicity}
    $\wp(u) = \wp(v)$.
\end{proof}


\subsection{Question 3 b).}
\noindent
Let $a_1, \cdots , a_n$ and $b_1, \cdots , b_m$ be complex numbers such that none of the numbers $\wp(a_i) - \wp(b_j)$
is zero. Let
\[
    f(z) = \prod_{i=1}^{n} [ \wp(z) - \wp(a_i) ] / \prod_{r=1}^{m} [ \wp(z) - \wp(b_r) ].
\]
Prove that $f$ is an even elliptic function with zeros at $a_1, \cdots , a_n$ and poles at $b_1, \cdots , b_m$.

\subsection*{Answer}
\noindent
We have to prove that $f$ is elliptic, that $f$ is even, that $f$ has zeros at $a_1, \cdots , a_n$ and that $f$
has poles at $b_1, \cdots , b_m$. We also need to explain the impact of any of the $\wp(a_i) - \wp(b_j)$ being zero.

\begin{proof}
    The sum, difference, product and quotient of elliptic functions are also elliptic functions are also elliptic
    functions, since the set of all elliptic functions for a fixed lattice is a field. Hence, $f(z)$ is elliptic.\\
    The function $f(z)$ is even because
    \[
        f(z) = \prod_{i=1}^{n} [ \wp(z) - \wp(a_i) ] / \prod_{r=1}^{m} [ \wp(z) - \wp(b_r) ] =
        \prod_{i=1}^{n} [ \wp(-z) - \wp(a_i) ] / \prod_{r=1}^{m} [ \wp(-z) - \wp(b_r) ] =
        f(-z).
    \]
    The function $f(z)$ has zeros at $a_1, \cdots , a_n$ since $\wp(a_i) - \wp(a_i) = 0$ and has poles at
    $b_1, \cdots , b_m$ since $\wp(b_j) - \wp(b_j) = 0$.\\
    If $\wp(a_i) - \wp(b_j) = 0$ for some $i,j$ then $f(z)$ might be undefined.
\end{proof}


\subsection{Question 4.}
\noindent
Prove that every even elliptic function $f$ is a rational function of $\wp$, where the periods of $\wp$ are a subset of
the periods of $f$.

\subsection*{Answer}
\noindent
We answer this by proving the following two propositions.

\begin{proposition}[1]
    Let $f$ be an even elliptic function, whose pole set is contained in $L$. Then $f$can be represented as a
    polynomial in $\wp$. The degree of this polynomial is half of the order of $f$.
\end{proposition}

\begin{proof}
    The Laurent series of $f$ in $0$ has only even coefficients, and is hence of the form
    \[
        f(z) = a_{-2n}z^{-2n}+a_{-2(n-1)}z^{-2(n-1)}+ \cdots \ \ \ \ n \geq 1, a_{-2n} \neq 0.
    \]
    The Laurent series of $\wp$ is of the form
    \[
        \wp(z) = z^{-2} + \cdots \ ,
    \]
    and for the n-th power we obtain
    \[
        \wp(z)^n = z^{-2n} + \cdots
    \]
    Just as $f$ and $\wp$ the function
    \[
        g(z) = f(z) - a_{-2n} \wp(z)^n
    \]
    is an even elliptic function whose pole set is contained in $L$. The order of $g$ is strictly smaller than the
    order of $f$. The proof now is obtained by induction.
\end{proof}


\begin{proposition}[2]
    The field of all even elliptic functions for the lattice $L$ is equal to $\mathbb{C}(\wp(z))$, as a subfield
    of $K(L)$, and is thus isomorphic to the field of rational functions.
\end{proposition}

\begin{proof}
    Let $f$ be a non-constant even elliptic function. If $a$ is a pole of $f$ which is not contained in $L$, the
    function
    \[
        z \mapsto (\wp(z)-\wp(a))^N \cdot f(z)
    \]
    has in $z=a$ a removable singularity, if $N$ is sufficiently large. Since $f$ has only finitely many poles mod $L$,
    we find finitely many points $a_1, \cdots , a_m$ and natural numbers $N_1, \cdots , N_m$ such that
    \[
        g(z) = f(z) \cdot \prod_{j=1}^{n} [ \wp(z) - \wp(a_j) ]^{N_j}
    \]
    has no poles outside $L$. By proposition 1 $g(z)$ is a polynomial in $\wp(z)$.
\end{proof}


\subsection{Question 5.}
\noindent
Prove that every elliptic function $f$ can be expressed in the form
\[
    f(z) = R_1(\wp(z))+ \wp'(z)R_2(\wp(z)),
\]
where $R_1, R_2$ are rational functions and $\wp$ has the same set of periods as $f$.

\subsection*{Answer}

\begin{proof}
    Let $f$ be an elliptic function, $E_i$ an even elliptic function and $O_i$ an odd elliptic function, all for the
    lattice $L$, then

    \begin{align*}
        f(z) &= \frac{f(z)+f(-z)}{2}+\frac{f(z)-f(-z)}{2} \\
        f(z) &= E_1(z)+O_1(z) \\
        f(z) &= E_1(z)+\wp'(z) \frac{O_1(z)}{\wp'(z)} \\
        f(z) &= E_1(z)+\wp'(z) E_2(z) \Rightarrow \\
        f(z) &= R_1(\wp(z))+ \wp'(z)R_2(\wp(z))
    \end{align*}
\end{proof}


\subsection{Question 6.}
\noindent
Let $f$ and $g$ be two elliptic functions with the same set of periods. Prove that there exists a polynomial $P(x, y)$,
not identically zero, such that
\[
    P(f(z), g(z) )= C
\]
where $C$ is a constant (depending on $f$ and $g$ but not on $z$).

\subsection*{Answer}
\noindent
We answer this by proving the following proposition.

\begin{proposition}
    $1=1$
\end{proposition}

\begin{proof}
    Assume that:

    \begin{align*}
        1 &= 1
    \end{align*}

\end{proof}


\subsection{Question 7.}
\noindent
The discriminant of the polynomial $f(x) = 4(x - x_1)(x - x_2)(x - x_3)$ is the product \\
$16((x_2 - x_1)(x_3 - x_2)(x_3 - x_1))^2$. Prove that the discriminant of $f(x) =  4x^3 - a x - b$ is $a^3 - 27b^2$.

\subsection*{Answer}
Clearly, the definition of this discriminant is equal to the definition of the polynomial discriminant divided by 16,
or $\Delta(p(x)) = \frac{1}{16} Disc(p(x))$. (By Mathematica) Discriminant$[4x^3 - a x - b, x]=16 (a^3 - 27 b^2)$, so
$\Delta(4x^3 - a x - b) =a^3 - 27b^2$.


\subsection{Question 8.}
\noindent
The differential equation for $\wp(z)$ shows that $\wp'(z)=0$ if $z=\omega_1 / 2$, $z=\omega_2 / 2$ or
$z=(\omega_1 + \omega_2) / 2$. Show that
\[
    \wp''(\frac{\omega_1}{2}) = 2 (e_1 - e_2)(e_1 - e_3)
\]
and obtain corresponding formulas for $\wp''(\frac{\omega_2}{2})$ and $\wp''(\frac{\omega_1 + \omega_2}{2})$.

\subsection*{Answer}
\noindent
From 1.9(5) we know that $\wp''(z)=6{\wp(z)}^2-\frac{1}{2}g_2$ and from\\
https://mathworld.wolfram.com/WeierstrassEllipticFunction.html (105) we know that\\
$g_2=-4(e_1 e_2 + e_1 e_3 + e_2 e_3)$. We apply this to $\frac{\omega_1}{2}$ and prove the following proposition.

\begin{proposition}
    $\wp''(\frac{\omega_1}{2})=6e_1^2-\frac{1}{2}(-4(e_1 e_2 + e_1 e_3 + e_2 e_3))=2 (e_1 - e_2)(e_1 - e_3).$
\end{proposition}

\begin{proof}

    \begin{align*}
        \wp''(\frac{\omega_1}{2}) &=6e_1^2-\frac{1}{2}(-4(e_1 e_2 + e_1 e_3 + e_2 e_3)) \\
        \wp''(\frac{\omega_1}{2}) &=6e_1^2+2(e_1 e_2 + e_1 e_3 + e_2 e_3) \\
        \wp''(\frac{\omega_1}{2}) &=6e_1^2+2e_1 e_2 + 2e_1 e_3 + 2e_2 e_3 \\
        \wp''(\frac{\omega_1}{2}) &=6e_1^2+2e_1 e_3 + 2e_2 (e_1 +  e_3) \\
        \wp''(\frac{\omega_1}{2}) &=6e_1^2+2e_1 e_3 - 2{e_2}^2 \\
        \wp''(\frac{\omega_1}{2}) &=6e_1^2-2{e_1}^2-2e_1(-e_1-e_3)- 2{e_2}^2 \\
        \wp''(\frac{\omega_1}{2}) &=6e_1^2-2{e_1}^2-2e_1e_2- 2{e_2}^2 \\
        \wp''(\frac{\omega_1}{2}) &=6e_1^2 + 2e_1 e_2 - 2{e_1}^2 -4 e_1 e_2 - 2{e_2}^2 \\
        \wp''(\frac{\omega_1}{2}) &=6e_1^2 + 2e_1 e_2 - 2(-e_1-e_2)^2 \\
        \wp''(\frac{\omega_1}{2}) &=4e_1^2 + 2e_1^2 + 2e_1 e_2 - 2(-e_1-e_2)^2 \\
        \wp''(\frac{\omega_1}{2}) &=4e_1^2 -2e_1 (-e_1-e_2) - 2{e_3}^2 \\
        \wp''(\frac{\omega_1}{2}) &=4e_1^2 -2e_1 e_3 - 2{e_3}^2 \\
        \wp''(\frac{\omega_1}{2}) &=2(2e_1+e_3)(e_1-e_3) \\
        \wp''(\frac{\omega_1}{2}) &=2(e_1-e_2)(e_1-e_3)
    \end{align*}

\end{proof}

Corresponding formulas for $\wp''(\frac{\omega_2}{2})$ and $\wp''(\frac{\omega_1 + \omega_2}{2})$ are:
\[
    \wp''(\frac{\omega_1+\omega_2}{2})=2 (e_2 - e_1)(e_2 - e_3),
\]
and
\[
    \wp''(\frac{\omega_2}{2})=2 (e_3 - e_1)(e_3 - e_2).
\]


\subsection{Question 9.}
\noindent
According to Exercise 4, the function $\wp(2z)$ is a rational function of $\wp(z)$. Prove that, in fact,
\[
    \wp(2z) = \frac{(\wp(z)^2 + \frac{1}{4}g_2)^2+2g_3\wp(z)}{4\wp(z)^3-g_2\wp(z)-g_3}.
\]

\subsection*{Answer}
\noindent
We start from the 'Weierstrass Doubling Formula' $\wp(2z)= \frac{1}{4}\left\{ \frac{\wp''(z)}{\wp'(z)}\right\}^2-2\wp(z)$.


\begin{proof}
    \begin{align*}
        \wp(2z) &= \frac{1}{4}\left\{ \frac{\wp''(z)}{\wp'(z)}\right\}^2-2\wp(z) \\
        \wp(2z) &= \frac{1}{4} \frac{(6\wp(z)-\frac{1}{2}g_2)^2}{4 \wp(z)^3 -g_2\wp(z)-g_3}-2\wp(z) \\
        \wp(2z) &= \frac{1}{4} \frac{(6\wp(z)-\frac{1}{2}g_2)^2-4 \cdot 2\wp(z)(4 \wp(z)^3 -g_2\wp(z)-g_3) }{4 \wp(z)^3 -g_2\wp(z)-g_3} \\
        \wp(2z) &= \frac{1}{4} \frac{36\wp(z)^4 - 6g_2 \wp(z)^2 + \frac{1}{4}g_2^2 -32\wp(z)^4 + 8g_2\wp(z)^2 + 8g_3\wp(z) }{4 \wp(z)^3 -g_2\wp(z)-g_3} \\
        \wp(2z) &= \frac{9\wp(z)^4 - \frac{3}{2}g_2 \wp(z)^2 + \frac{1}{16}g_2^2 -8\wp(z)^4 + 2g_2\wp(z)^2 + 2g_3\wp(z) }{4 \wp(z)^3 -g_2\wp(z)-g_3} \\
        \wp(2z) &= \frac{\wp(z)^4 + \frac{1}{2}g_2 \wp(z)^2 + \frac{1}{16}g_2^2 + 2g_3\wp(z) }{4 \wp(z)^3 -g_2\wp(z)-g_3} \\
        \wp(2z) &= \frac{(\wp(z)^2 + \frac{1}{4}g_2)^2+2g_3\wp(z)}{4\wp(z)^3-g_2\wp(z)-g_3}.
    \end{align*}
\end{proof}


\subsection{Question 10.}
\noindent
Let $\omega_1$ and $\omega_2$ be complex numbers with non-real ratio. Let $f(z)$ be an entire function and
assume that there are constants $a$ and $b$ such that $f(z+\omega_1)=a f(z), f(z+\omega_2)=b f(z)$, for all $z$.
Show that $f(z) = A e^{Bz}$, where $A$ and $B$ are constants.

\subsection*{Answer}
\noindent
We answer this by proving the following proposition.

\begin{proposition}
    $1=1$
\end{proposition}

\begin{proof}
    Assume that:

    \begin{align*}
        1 &= 1
    \end{align*}

\end{proof}


\subsection{Question 11.}
\noindent
If $k \geq 2$ and $\tau \in H$ prove that the Eisenstein series
\[
    G_{2k}(\tau) = \sum_{m,n}{'}(m + n\tau)^{-2k}
\]
has the Fourier expansion
\[
    G_{2k}(\tau) = 2\zeta(2k) +\frac{2(2 \pi i)^{2k}}{(2k-1)!}\sum_{n=1}^{\infty}\sigma_{2k-1}(n)e^{2\pi i n \tau}.
\]

\subsection*{Answer}
\noindent
We answer this by starting from the known identity
$
    \frac{1}{z} + \sum_m{'} ( \frac{1}{z+m} - \frac{1}{m} ) = \pi \cot(\pi z)
$
and work our way from there.

\begin{proof}
    \begin{align*}
        \frac{1}{z} + \sum_m{'} ( \frac{1}{z+m} - \frac{1}{m} ) &= \pi \cot(\pi z) &&(1)\\
        \frac{1}{z} + \sum_m{'} ( \frac{1}{z+m} - \frac{1}{m} ) &= \pi i -2 \pi i \sum_{n=1}^\infty e^{2 \pi i z n} &&(2)\\
        (-1)^j j! \sum_{m=-\infty}^{\infty} \frac{1}{(z+m)^{j+1}} &= -(2 \pi i)^{j+1} \sum_{n=1}^\infty n^j e^{2 \pi i z n} &&(3)\\
        (-1)^j j! \sum_{m=-\infty}^{\infty} \frac{1}{(m+N\tau)^{j+1}} &= -(2 \pi i)^{j+1} \sum_{n=1}^\infty n^j e^{(2 \pi i N\tau) n} &&(4)\\
        (-1)^j j! \sum_{N=1}^{\infty} \sum_{m=-\infty}^{\infty} \frac{1}{(m+N\tau)^{j+1}} &= -(2 \pi i)^{j+1} \sum_{n=1}^\infty n^j \frac{e^{(2 \pi i \tau)n}}{1-e^{(2 \pi i \tau)n}} &&(5)\\
        \frac{1}{2} \sum_{m,N}{'} \frac{1}{(m+N\tau)^{2k}} - \sum_{m=1}^\infty \frac{1}{m^{2k}} &= \frac{(2 \pi i)^{2k}}{(2k-1)!} \sum_{n=1}^\infty n^{2k-1} \frac{e^{(2 \pi i \tau)n}}{1-e^{(2 \pi i \tau)n}} &&(6)\\
        \sum_{m,N}{'} \frac{1}{(m+N\tau)^{2k}} &= 2\zeta(2k) + 2 \frac{(2 \pi i)^{2k}}{(2k-1)!} \sum_{n=1}^\infty n^{2k-1} \frac{e^{(2 \pi i \tau)n}}{1-e^{(2 \pi i \tau)n}} &&(7)\\
        G_{2k}(\tau) &= 2\zeta(2k) +\frac{2(2 \pi i)^{2k}}{(2k-1)!}\sum_{n=1}^{\infty}\sigma_{2k-1}(n)e^{2\pi i n \tau} &&(8)
    \end{align*}

\end{proof}

\subsection*{Remarks}
\noindent
(1): Starting point.\\
(2): Rework $\pi \cot(\pi z) $. \\
(3): Differentiate both sides over $z$, \  $j$ times. \\
(4): Replace $z$ by $N\tau$. \\
(5): Sum both sides over $N$ from $1$ to $\infty$.\\
(6): Now, assume $j$ is odd, so $j=2k-1$. \\
(7): Multiply both sides by $2$ and replace $\sum_{m=1}^\infty \frac{1}{m^{2k}}$ by $\zeta(2k)$.\\
(8): Apply that $\sum_{n=1}^\infty n^{2k-1} \frac{e^{(2 \pi i \tau)n}}{1-e^{(2 \pi i \tau)n}} = \sum_{n=1}^{\infty}\sigma_{2k-1}(n)e^{2\pi i n \tau}$.

\subsection{Question 12.}
\noindent
Refer to exercise 11. If $\tau \in H$ prove that
\[
    G_{2k}(-\frac{1}{\tau}) = \tau^{2k} G_{2k}(\tau).
\]

\subsection*{Answer}

\begin{proof}
    \begin{align*}
        G_{2k}(-\frac{1}{\tau}) &= \sum_{m,n}{'} \frac{1}{(m + n (\frac{-1}{\tau}))^{2k}} \\
        G_{2k}(-\frac{1}{\tau}) &= \sum_{m,n}{'} \frac{1}{(m - \frac{n}{\tau})^{2k}} \\
        G_{2k}(-\frac{1}{\tau}) &= \sum_{m,n}{'} \tau^{2k} \frac{1}{(m\tau - n)^{2k}} \\
        G_{2k}(-\frac{1}{\tau}) &= \tau^{2k} \sum_{m,n}{'} \frac{1}{(m\tau - n)^{2k}} \\
        G_{2k}(-\frac{1}{\tau}) &= \tau^{2k} \sum_{m,n}{'} \frac{1}{(-n +m\tau)^{2k}} \\
        G_{2k}(-\frac{1}{\tau}) &= \tau^{2k} \sum_{m,n}{'} \frac{1}{(n + m\tau)^{2k}} \\
        G_{2k}(-\frac{1}{\tau}) &= \tau^{2k} \sum_{m,n}{'} \frac{1}{(m + n\tau)^{2k}} \\
        G_{2k}(-\frac{1}{\tau}) &= \tau^{2k} \sum_{m,n}{'} G_{2k}(\tau)
    \end{align*}

\end{proof}


\subsection{Question 12 a).}
\noindent
Refer to exercise 11. If $\tau \in H$ deduce that
\[
    G_{2k}(\frac{i}{2}) = (-4)^{k}G_{2k}(2i).
\]

\subsection*{Answer}
\noindent
We answer this by using
$
    G_{2k}(-\frac{1}{\tau}) = \tau^{2k} \sum_{m,n}{'} G_{2k}(\tau)
$
and set $\tau = 2i$.

\begin{proof}
    \begin{align*}
        G_{2k}(\frac{i}{2}) &= (2i)^{2k} G_{2k}(2i) \\
        &= (-4)^{k} G_{2k}(2i)
    \end{align*}
\end{proof}


\subsection{Question 12 b).}
\noindent
Refer to exercise 11. If $\tau \in H$ deduce that
\[
    G_{2k}(i) =0 , \text{if k is odd.}
\]

\subsection*{Answer}
\noindent
We answer this by using
$
G_{2k}(-\frac{1}{\tau}) = \tau^{2k} \sum_{m,n}{'} G_{2k}(\tau),
$
set $\tau = i$, $k$ odd. Note that $\frac{-1}{i}=i$.

\begin{proof}
    \begin{align*}
        G_{2k}(i) &= (i)^{2k} G_{2k}(i) \\
        G_{2k}(i) &= (-1)^{k} G_{2k}(i) \\
        G_{2k}(i) &= 0
    \end{align*}
\end{proof}


\subsection{Question 12 c).}
\noindent
Refer to exercise 11. If $\tau \in H$ deduce that
\[
    G_{2k}(e^{2\pi i/3}) =0 , \text{if } k \not\equiv 0\ (\textrm{mod}\ 3).
\]

\subsection*{Answer}
\noindent
We answer this by using
$
G_{2k}(-\frac{1}{\tau}) = \tau^{2k} \sum_{m,n}{'} G_{2k}(\tau)
$
and set $\tau = e^{\frac{1}{3}\pi i}$. Note that $\frac{-1}{e^{\frac{1}{3}\pi i}}=e^{\frac{2}{3}\pi i}$.

\begin{proof}
    \begin{align*}
        G_{2k}(e^{\frac{2}{3}\pi i}) &= (e^{\frac{1}{3}\pi i})^{2k} G_{2k}(e^{\frac{1}{3}\pi i}) \\
        G_{2k}(e^{\frac{2}{3}\pi i} + 1) &= (e^{\frac{1}{3}\pi i})^{2k} G_{2k}(e^{\frac{1}{3}\pi i}) \\
        G_{2k}(e^{\frac{1}{3}\pi i}) &= (e^{\frac{1}{3}\pi i})^{2k} G_{2k}(e^{\frac{1}{3}\pi i}) \\
        G_{2k}(e^{\frac{1}{3}\pi i}) &= (e^{\frac{2}{3}\pi i})^{k} G_{2k}(e^{\frac{1}{3}\pi i}) \\
        G_{2k}(e^{\frac{1}{3}\pi i}) &=
            \begin{cases}
                     (e^{\frac{2}{3}\pi i}) G_{2k}(e^{\frac{1}{3}\pi i}) & \text{if } k \equiv 1 \mod{3},\\
                     (e^{\frac{4}{3}\pi i}) G_{2k}(e^{\frac{1}{3}\pi i}) & \text{if } k \equiv 2 \mod{3},\\
                     G_{2k}(e^{\frac{1}{3}\pi i}) & \text{if } k \equiv 0 \mod{3}\\
            \end{cases}
    \end{align*}

The cases $k=1,2$ can only occur if $G_{2k}(e^{\frac{1}{3}\pi i}) = 0$.
\end{proof}


\subsection{Question 13.}
\noindent
Ramanujan's tau function $\tau(n)$ is defined by the Fourier expansion
\[
    \Delta(\tau)=(2\pi)^{12}\sum_{n=1}^{\infty}\tau(n)e^{2\pi i n \tau},
\]
derived in Theorem 2.19. Prove that
\[
    \tau(n)=8000 \{ (\sigma_3 \circ \sigma_3) \circ \sigma_3 \}(n) - 147 (\sigma_5 \circ \sigma_5)(n),
\]
where $f \circ g$ denotes the Cauchy product of two sequences,
\[
    (f \circ g)(n) = \sum_{k=0}^{n}f(k)g(n-k),
\]
and
\[
    \sigma_{\alpha}(n)= \sum_{d/n} d^{\alpha} \text{ for } n\geq 1, \text{ with } \sigma_3(0)=\frac{1}{240},
    \sigma_5(0)=-\frac{1}{504}.
\]
(Hint: Theorem 1.18.)

\subsection*{Answer}
\noindent
We answer this by using the following identities:
\[
    \Delta(\tau)={g_2(\tau)}^3-27{g_3(\tau)}^2
\]
\[
    {g_2(\tau)}=\frac{4}{3}\pi^4 \cdot 240 \cdot \sum_{n=0}^{\infty}\sigma_3(n)e^{2\pi i n \tau}
\]
\[
    {g_3(\tau)}=-\frac{8}{27}\pi^6 \cdot 504 \cdot \sum_{n=0}^{\infty}\sigma_5(n)e^{2\pi i n \tau}
\]

\begin{proof}
    \begin{align*}
        \Delta(\tau)&={g_2(\tau)}^3-27{g_3(\tau)}^2 \\
        \Delta(\tau)&={(\frac{4}{3}\pi^4 \cdot 240 \cdot \sum_{n=0}^{\infty}\sigma_3(n)e^{2\pi i n \tau})}^3-27 \cdot {(-\frac{8}{27}\pi^6 \cdot 504 \cdot \sum_{n=0}^{\infty}\sigma_5(n)e^{2\pi i n \tau})}^2 \\
        \Delta(\tau)&=32768000\pi^{12} \sum_{n=0}^{\infty}\sigma_3 \circ \sigma_3 \circ \sigma_3(n)e^{2\pi i n \tau} - 602112\pi^{12} \sum_{n=0}^{\infty}\sigma_5 \circ \sigma_5(n)e^{2\pi i n \tau} \\
        \Delta(\tau)&=(2\pi)^{12}\sum_{n=0}^{\infty} (8000\sigma_3 \circ \sigma_3 \circ \sigma_3(n)-147\sigma_5 \circ \sigma_5(n))e^{2\pi i n \tau}
    \end{align*}
\end{proof}


\subsection{Question 14.}
\noindent
A series of the form $\sum_{n=1}^{\infty}f(n)\frac{x^n}{1-x^n}$ is called a Lambert series. Assuming absolute
convergence, prove that:
\[
    \sum_{n=1}^{\infty}f(n) \frac{x^n}{1-x^n} = \sum_{n=1}^{\infty} F(n)x^n,
\]
where $F(n) = \sum_{d/n}f(d)$.

\subsection*{Answer}
\noindent
We answer this by using the following identity:
\[
    \frac{x^n}{1-x^n} = \frac{1}{1-x^n}-1 = x^n+x^{2n}+x^{3n}+ \cdots
\]

\begin{proof}
    \[
        \renewcommand*{\arraystretch}{1.5}
        \setlength{\arraycolsep}{0pt}
        \begin{array}{
            r
            *{9}{ >{{}}c<{{}} r }
        }
            \sum_{n=1}^{\infty}f(n) \frac{x^n}{1-x^n} &=& f(1)x &+& f(1)x^2 &+& f(1)x^3 &+& f(1)x^4 &+& f(1)x^5 &+& f(1)x^6 &+& f(1)x^7 &+& f(1)x^8 &+& \dotsb \\
            & &      &+& f(2)x^2 & &      &+& f(2)x^4 & &      &+& f(2)x^6 & &      &+& f(2)x^8 &+& \dotsb \\
            & &      & &      &+& f(3)x^3 & &      & &      &+& f(3)x^6 & &      & &      &+& \dotsb \\
            & &      & &      & &         & &      & &      & &      & &      &+& f(4)x^8 &+& \dotsb \\
        \end{array}
    \]
    \begin{align*}
        \sum_{n=1}^{\infty}f(n) \frac{x^n}{1-x^n} &= f(1)x
        + ( f(1) + f(2) )x^2
        + ( f(1) + f(3) )x^3
        + ( f(1) + f(2) + f(4) )x^4 + \dotsb\\
        &= \sum_{n=1}^{\infty} \sum_{d/n} f(d) x^n \\
        &= \sum_{n=1}^{\infty} F(n) x^n
    \end{align*}
\end{proof}


\subsection{Question 14 a).}
\noindent
Apply the result of exercise 14 to obtain the following formula valid for $|x|<1$.
\[
    \sum_{n=1}^{\infty} \frac{\mu(n) x^n}{1-x^n} = x.
\]

\subsection*{Answer}
\noindent
We answer this by using the following identity:
\[
    \sum_{d/n}\mu(d) = I(n) =
        \begin{cases}
                                              1 \text{ if } n = 1,\\
                                              0 \text{ if } n > 1.
        \end{cases}
\]

\begin{proof}
    \begin{align*}
        \sum_{n=1}^{\infty}\mu(n) \frac{x^n}{1-x^n} &= \sum_{n=1}^{\infty} \sum_{d/n} \mu(d) x^n \\
        &= \sum_{n=1}^{\infty} I(n) x^n \\
        &= x
    \end{align*}

\end{proof}


\subsection{Question 14 b).}
\noindent
Apply the result of exercise 14 to obtain the following formula valid for $|x|<1$.
\[
    \sum_{n=1}^{\infty} \frac{\varphi(n) x^n}{1-x^n} = \frac{x}{(1-x)^2}.
\]

\subsection*{Answer}
\noindent
We answer this by using the following identities:
\[
    \sum_{d/n}\varphi(d) = n
\]
\[
    \sum_{n=1}^{\infty} n x^n = \frac{x}{(1-x)^2}
\]

\begin{proof}
    \begin{align*}
        \sum_{n=1}^{\infty}\varphi(n) \frac{x^n}{1-x^n} &= \sum_{n=1}^{\infty} \sum_{d/n} \varphi(d) x^n \\
        &= \sum_{n=1}^{\infty} n x^n \\
        &= \frac{x}{(1-x)^2}
    \end{align*}
\end{proof}


\subsection{Question 14 c).}
\noindent
Apply the result of exercise 14 to obtain the following formula valid for $|x|<1$.
\[
    \sum_{n=1}^{\infty} \frac{n^\alpha x^n}{1-x^n} = \sum_{n=1}^{\infty} \sigma_{\alpha}(n)x^n.
\]

\subsection*{Answer}
\noindent
We answer this by using the following identities:
\[
    \sum_{d/n}d^\alpha = \sigma_{\alpha}(n)
\]

\begin{proposition}
    $1=1$
\end{proposition}

\begin{proof}
    Assume that:

    \begin{align*}
        1 &= 1
    \end{align*}

\end{proof}


\subsection{Question 14 d).}
\noindent
Apply the result of exercise 14 to obtain the following formula valid for $|x|<1$.
\[
    \sum_{n=1}^{\infty} \frac{\lambda(n) x^n}{1-x^n} = \sum_{n=1}^{\infty} x^{n^2}.
\]

\subsection*{Answer}
\noindent
We answer this by proving the following proposition.

\begin{proposition}
    $1=1$
\end{proposition}

\begin{proof}
    Assume that:

    \begin{align*}
        1 &= 1
    \end{align*}

\end{proof}


\subsection{Question 14 e).}
\noindent
Apply the result of exercise 14 c) to express $g_2(\tau)$ and $g_3(\tau)$ in terms of Lambert series in
$x=e^{2\pi i \tau}$.

\subsection*{Answer}
\noindent
We answer this by proving the following proposition.

\begin{proposition}
    $1=1$
\end{proposition}

\begin{proof}
    Assume that:

    \begin{align*}
        1 &= 1
    \end{align*}

\end{proof}


\subsection{Question 15 a).}
\noindent
Let
\[
    G(x)=\sum_{n=1}^{\infty} \frac{n^5 x^n}{1-x^n},
\]
and
\[
    F(x)=\sum_{\substack{n=1\\ \text{n odd}}}^{\infty} \frac{n^5 x^n}{1-x^n}.
\]
Show that
\[
    F(x)=G(x)-34G(x^2)+64G(x^4).
\]

\subsection*{Answer}
\noindent
We answer this by proving the following proposition.

\begin{proposition}
    $1=1$
\end{proposition}

\begin{proof}
    Assume that:

    \begin{align*}
        1 &= 1
    \end{align*}

\end{proof}


\subsection{Question 15 b).}
\noindent
Show that
\[
    F(x)=\sum_{\substack{n=1\\ \text{n odd}}}^{\infty} \frac{n^5}{1+e^{n\pi}} = \frac{31}{504}.
\]

\subsection*{Answer}
\noindent
We answer this by proving the following proposition.

\begin{proposition}
    $1=1$
\end{proposition}

\begin{proof}
    Assume that:

    \begin{align*}
        1 &= 1
    \end{align*}

\end{proof}