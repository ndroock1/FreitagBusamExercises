
\section{Euclid}

A prime number is a positive integer not equal to $1$ which is divisible only by $1$ and itself.\newline

The sequence of prime numbers starts with
$$2,3,5,7,11,13,17,19,23,29,31,37,41,43,47,53,59,61,67,71,73,79,83,89,97, \ldots$$

The Fundamental Theorem of Arithmetic claims that every integer greater than 1 either is a prime number itself or can be
represented as the product of prime numbers and that, moreover, this representation is unique, up to (except for) the
order of the factors. So, if
$$p_1=2, p_2=3, p_3=5, \ldots$$
and in general $p_n$ is equal to the $n$-th prime number and $a>1$ is a positive integer, then
$$a=p_{\alpha_1}^{b_1} \ldots p_{\alpha_\nu}^{b_\nu}$$
where $\nu \geq 1, 1 \leq \alpha_1 < \alpha_2 \ldots < \alpha_\nu$ and the sum of the exponents $b_i \geq 1$, is the
unique representation of $a$.

