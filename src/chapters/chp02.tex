\section{Liouville's Theorems}

\begin{definition}[Elliptic Function]
    Elliptic functions are doubly periodic meromorphic functions on C.
\end{definition}

\begin{theorem}[First Liouville Theorem]
    Any elliptic function without poles is constant.
\end{theorem}
\begin{proof}
    The set
    \[F = F(\omega_1, \omega_2) = \{ t_1 \omega_1 + t_2 \omega_2 ; 0 \leq t1, t2 \leq 1 \}\]
    is called a fundamental region for a lattice L with respect to the basis $\omega_1, \omega_2$.
    For any point $z \in \mathbb{C}$ there exists a lattice point $\omega \in L$, such that
    $z-\omega \in F$. Any value of an elliptic function is hence also taken in a fundamental region $F$.
    But $F$ is bounded and closed in $\mathbb{C}$, hence, any continuous function defined on $F$ is bounded.
    An elliptic function without poles is thus bounded on F, and hence also on the entire $\mathbb{C}$,
    so it is a constant function.
\end{proof}

\begin{theorem}[Second Liouville Theorem]
    An elliptic function has only finitely many poles modulo $L$ (i.e. on the torus $\mathbb{C} / L$), and the
    sum of their residues vanishes:
    \[ \sum_z Res(f; z)=0\]
    Here, the sum is taken over a system of representatives modulo $L$ for all poles of f.
\end{theorem}
\begin{proof}
    TBD.
\end{proof}