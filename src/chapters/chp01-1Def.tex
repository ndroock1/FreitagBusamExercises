\section{Definitions}

\begin{definition}[doubly periodic]
    A function $f$ is called doubly periodic if it has two periods $\omega_1$ and $\omega_2$ whose ratio
    $\omega_2 / \omega_1$ is not real.
\end{definition}

\begin{definition}[fundamental pair]
    Let f have two periods $\omega_1$ and $\omega_2$ whose ratio $\omega_2 / \omega_1$ is not real.
    The pair $(\omega_1, \omega_2)$ is called a fundamental pair if every
    period of $f$ is of the form $m \cdot \omega_1 + n \cdot \omega_2$, where $m$ and $n$ are integers.
\end{definition}

\begin{definition}[equivalent pairs]
    Two pairs of complex numbers $(\omega_1, \omega_2)$ and $({\omega_1}^\prime, {\omega_2}^\prime)$, each with
    non-real ratio, are called equivalent if they generate the same lattice of periods; that is,
    if $\Omega(\omega_1, \omega_2) = \Omega({\omega_1}^\prime, {\omega_2}^\prime)$.
\end{definition}

\begin{definition}[elliptic function]
    A function $f$ is called elliptic if it has the following two properties :\\
    (a) $f$ is doubly periodic.\\
    (b) $f$ is meromorphic (its only singularities in the finite plane are poles).
\end{definition}

\begin{definition}[Weierstrass $\wp$ function]
    The Weierstrass $\wp$ function is defined by the series:
    \[
        \wp(z)= \frac{1}{z} + \sum_{\omega}{'} ( \frac{1}{(z - \omega)^2} - \frac{1}{z^2} ).
    \]
\end{definition}

\begin{definition}[Eisenstein series]
    If $n > 3$, then the series:
    \[
        G_n = \sum_{\omega}{'} \frac{1}{\omega^n}
    \]
    is called the Eisenstein series of order $n$.
\end{definition}

\begin{definition}[invariants $g_2, g_3$]
    The invariants $g_2$ and $g_3$ are the numbers defined by the relations:
    \[
        g_2 = 60 G_4
    \]
    \[
        g_3 = 140 G_6.
    \]
\end{definition}

\begin{definition}[differential equation $\wp$]
    The differential equation for $\wp$ takes the form:
    \[
        \wp'(z)^2 = 4 \wp(z)^3 - g_2 \wp(z) - g_3.
    \]
\end{definition}

\begin{definition}[$e_1, e_2, e_3$]
    We denote by $e_1, e_2, e_3$ the values of $\wp$ at the half-periods
    \[
        e_1 = \wp(\frac{\omega_1}{2}), e_2 = \wp(\frac{\omega_2}{2}), e_3 = \wp(\frac{\omega_1 + \omega_2}{2}).
    \]
\end{definition}

\begin{definition}[Klein's modular function $J(\tau)$]
    If $\omega_2 / \omega_1$ is not real we define
    \[
        J(\omega_1, \omega_2) = \frac{g_2(\omega_1, \omega_2)^3}{\Delta(\omega_1, \omega_2)}
    \]
    We write $J(\tau)$ for $J(1,\tau)$.
\end{definition}