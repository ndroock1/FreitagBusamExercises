\section{Complex Integration}


\begin{definition}[Curve (II.1.1)]
    \label{sec:Curve}
    A \textbf{curve} is a continuous map
    $$\alpha: [a,b] \rightarrow \mathbb{C}, a < b$$
    from a compact real interval into the complex plane. We call $\alpha(a)$ the starting point,
    and $\alpha(b)$ the end point of $\alpha$.
\end{definition}


\begin{definition}[Smooth Curve (II.1.2)]
    \label{sec:SmoothCurve}
    A \hyperref[sec:Curve]{curve} is called \textbf{smooth}, if it is continuously differentiable.
\end{definition}


\begin{definition}[Piecewise Smooth Curve (II.1.3)]
    \label{sec:PiecewiseSmoothCurve}
    Let\\
    TBD II.1.3
\end{definition}


\begin{definition}[Contour Integral (II.1.4)]
    \label{sec:ContourIntegral}
    Let
    $$\alpha : [a, b] \rightarrow \mathbb{C}$$
    be \hyperref[sec:SmoothCurve]{a smooth curve} and
    $$f: D \rightarrow \mathbb{C}, D \subset \mathbb{C},$$
    a continuous funtion, whose domain of definition contains the image of the curve $\alpha$,
    i.e. $D \supset \alpha([a,b]).$ Then one defines
    $$ \int_\alpha f:= \int_\alpha f(\zeta) d\zeta := \int_a^b f(\alpha(t))\alpha'(t)dt,$$
    and calls this complex number the line integral or \textbf{contour integral} of $f$ along $\alpha$.
    By the arc length of a smooth curve we mean
    $$l(\alpha):=\int_a^b |\alpha'(t)|dt.$$
\end{definition}


\begin{definition}[Properties Contour Integral (II.1.5)]
    \label{sec:PropContourIntegral}
    Let\\
    TBD
\end{definition}


\begin{definition}[Arcwise Connected Set (II.2.1)]
    \label{sec:ArcwiseConnected}
    Let\\
    TBD
\end{definition}


\begin{definition}[Domain (II.2.3)]
    \label{sec:Domain}
    By a \textbf{domain} we understand an \hyperref[sec:ArcwiseConnected]{arcwise connected} non-empty open set $D \subset \mathbb{C}$.
\end{definition}


\begin{definition}[A Star-shaped Domain (II.2.6)]
    \label{sec:StarDomain}
    A \textbf{star-shaped domain} is an open set $D \subset \mathbb{C}$ with the following property:
    There is a point $z_\star \in D$ such that for each point $z \in D$ the whole line segment joining $z_\star$ and $z$ is
    contained in $D$:
    $$\{ z_\star + t(z - z_\star) ; t \in [0, 1] \} \subset D .$$
    The point $z_\star$ is not uniquely determined, and is called a (possible) star center.
\end{definition}


\begin{definition}[Elementary Domain (II.2.8)]
    \label{sec:ElemDomain}
    A domain $D \subset \mathbb{C}$ is called an \textbf{elementary domain}, if
    any analytic function defined on $D$ has a primitive in $D$.
\end{definition}


\begin{definition}[Entire Function (II.3.6)]
    \label{sec:EntireFunction}
    An analytic function $f : \mathbb{C} \rightarrow \mathbb{C}$ is said to be \textbf{entire}.
\end{definition}