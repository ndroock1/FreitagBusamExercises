\section{Complex Integration}


\begin{theorem}[Closed Contour Theorem (II.1.6)]
    \label{sec:ClosedContourT}
    If a continuous function
    $$ f : D \rightarrow \mathbb{C}, \qquad D \subset \mathbb{C} \quad \text{open},$$
    \textbf{has a primitive} then
    $$\oint_\alpha f(\zeta)d\zeta = 0$$
    for any piecewise smooth curve $\alpha$ in $\mathbb{C}$.
\end{theorem}


\begin{theorem}[Main Theorem Calculus (II.2.4)]
    \label{sec:MainTCalculus}
    For a continuous function
    $$ f : D \rightarrow \mathbb{C}, \qquad D \subset \mathbb{C} \quad \text{a domain},$$
    the following three statements are equivalent:
    \begin{enumerate}[label=\alph*)]
        \item $f$ has a primitive.
        \item The integral of $f$ along any closed curve in $D$ vanishes.
        \item The integral of $f$ along any curve in $D$ depends only on the beginning and end points of the curve.
    \end{enumerate}
\end{theorem}


\begin{theorem}[Cauchy Integral Theorem for triangular paths (II.2.5)]
    \label{sec:CauchyITT}
    Let
    $$ f : D \rightarrow \mathbb{C}, \qquad D \subset \mathbb{C} \quad \text{open}$$
    be an analytic function (i.e. complex differentiable at any point $z \in D$). Let $z_1, z_2, z_3$ be three points
    in $D$ such that the triangle they span is also contained in D; then
    $$\int_{<z_1,z_2,z_3>} f(\zeta)d\zeta = 0.$$
\end{theorem}


\begin{theorem}[Cauchy Integral Theorem for star domains (II.2.7)]
    \label{sec:CauchyITR}
    Let \\
    TBD
\end{theorem}


\begin{theorem}[Analytic branch of the logarithm of f (II.2.9)]
    \label{sec:BranchLog}
    Let \\
    TBD
\end{theorem}


\begin{theorem}[Lemma (II.3.1)]
    \label{sec:LemmaII31}
    Let \\
    TBD
\end{theorem}


\begin{theorem}[Cauchy Integral Formula (II.3.1)]
    \label{sec:CauchyIF}
    Let \\
    TBD
\end{theorem}


\begin{theorem}[Leibniz' Rule (II.3.3)]
    \label{sec:LeibnizRule}
    Let \\
    TBD
\end{theorem}


\begin{theorem}[Generalized Cauchy Integral Formula (II.3.4)]
    \label{sec:GCauchyIF}
    Let \\
    TBD
\end{theorem}


\begin{theorem}[Morera's Theorem (II.3.5)]
    \label{sec:MoreraT}
    Let \\
    TBD
\end{theorem}


\begin{theorem}[Liouville's Theorem]
    \label{sec:LiouvilleT}
    Let \\
    TBD
\end{theorem}


\begin{theorem}[Fundamental Theorem of Algebra]
    \label{sec:FTAlgebra}
    Let \\
    TBD
\end{theorem}