\documentclass{book}
%\usepackage{mathmag}

\usepackage{amsmath,amsthm}
\usepackage{graphicx}
\usepackage{hyperref}
\usepackage{url}
\usepackage{amsfonts}

\usepackage{multicol}


% NOTE mathmag.sty calls the text fonts. For this template we are using times.sty
% from the standard LaTeX distribution.

%% IF YOU HAVE FONTS INSTALLED you can use these math fonts to more
%% closely approximate the final product.
%\usepackage{mtpro2}
%\usepackage{mathtime}

\theoremstyle{theorem}
\newtheorem{theorem}{Theorem}

\theoremstyle{definition}
\newtheorem*{definition}{Definition}
\newtheorem*{remark}{Remark}

\allowdisplaybreaks

\makeatletter
\@addtoreset{footnote}{page}
\makeatother

%%%%%%%%%%%%%%%%%%%%%%%%%%%%%%%%%%%%%%%%%%%%%%%%%%
\begin{document}


    \title{The Math Mag Article Template}

    \author{Author Name\\               %%%% Leave ALL of these as is in your initial submission
        \scriptsize affiliation line 1\\    %%%% to allow for double blind reviewing.
        affiliation line 2\\                %%%% They should be filled in when you are submitting
        email address}                      %%%% your final manuscript.

    \maketitle

    \noindent  Mathematics Magazine aims to provide lively and appealing mathematical exposition. The Magazine is not a research journal, so the terse style appropriate for such a journal (lemma-theorem-proof-corollary) is not appropriate for the Magazine. Articles should include examples, applications, historical background, and illustrations, where appropriate. They should be attractive and accessible to undergraduates and would, ideally, be helpful in supplementing undergraduate courses or in stimulating student investigations. Manuscripts on history are especially welcome, as are those showing relationships among various branches of mathematics and between mathematics and other disciplines.

    Submissions of articles are required via the Mathematics Magazine's Editorial Manager System. The name(s) of the author(s) should not appear in the file. Initial submissions in pdf or LaTeX form can be sent to the editor at \href{http://www.editorialmanager.com/mathmag/}{\url{www.editorialmanager.com/mathmag/}}.
    The Editorial Manager System will cue the author for all required information concerning the paper. Questions concerning submission of papers can be addressed to the editor at \href{mathmag@maa.org}{\url{mathmag@maa.org}}.

    \section{Mathematics Magazine style}

    The \textit{Mathematics Magazine} style incorporates the following \LaTeX\ packages.  These styles should \textit{not} be included in the document header.
    \begin{itemize}
        \item times
        \item pifont
        \item graphicx
        \item color
        \item AMS styles: amsmath, amsthm, amsfonts, amssymb
        \item url
    \end{itemize}
    Use of other \LaTeX\ packages should be minimized as much as possible. Math notation, like $c = \sqrt{a^2 +b^2}$, can be left in \TeX's default Computer Modern typefaces for manuscript preparation; or, if you have the appropriate fonts installed, the \texttt{mathtime} or \texttt{mtpro} packages may be used, which will better approximate the finished article.

    Web links can be embedded using the \verb~\url{...}~ command, which will result in something like \url{http://www.maa.org}.  These links will be active and stylized in the online publication.

    \section{First-level section heading}

    Section headings use an initial capital letter on the first word, with subsequent words lowercase.  In general, the style of the journal is to leave all section headings unnumbered.  Consult the journal editor if you wish to depart from this and other conventions.

    \subsection{Second-level heading}

    The same goes for second-level headings.  It is not necessary to add font commands to make the math within heads bold and sans serif; this change will occur automatically when the production style is applied.

    \section{Graphics and tables}

    Table for  \textit{Math Mag} should be set in an ``open" style: rules above and below the heading and a rule to end the table.  Note the use of \verb~\abrule~ and \verb~\brule~ to improve spacing in the table.

%    \begin{table}[h]
%        \begin{center}
%            \begin{tabular}{ccc}
%                \hline
%                Under  & $\pi(x) = \#\{\text{primes} \le x\}=$ &   $=\text{Li}(x)\pm$  Error\abrule\\
%                \hline
%                $500000$  &  41556 &  $41606.4 - 50.4$ \abrule \\
%                $1000000$ &  78501 &  $79627.5 - 126.5$\brule \\
%                $1500000$ & 114112 & $114263.1 - 151.1$\brule \\
%                $2000000$ & 148883 & $149054.8 - 171.8$\brule \\
%                $2500000$ & 183016 & $183245.0 - 229.0$\brule \\
%                $3000000$ & 216745 & $216970.6 - 225.6 $\brule \\
%                \hline
%            \end{tabular}
%        \end{center}
%        \caption{Sample table}
%    \end{table}


    Figures for  \textit{Math Mag} can be submitted as either color or black \& white graphics.  Generally, color graphics will be used for the online publication, and converted to black \& white images for the print journal.  We recommend using whatever graphics program you are most comfortable with, so long as the submitted graphic is provided as a separate file using a standard file format.

    For best results, please follow the following guidelines:
    \begin{enumerate}
        \item Bitmapped file formats---preferably TIFF or JPEG, but not BMP---are appropriate for photographs, using a resolution of at least 300 dpi at the final scaled size of the image.
        \item Line art will reproduce best if provided in vector form, preferably EPS. The thinnest line weight should be .5 pt.  Labels on a figure should be 9 pt in the same font style (italic, bold, etc.) as in the text.
        \item Alternatively, both photographs and line art can be provided as PDF files.  Note that creating a PDF does not affect whether the graphic is a bitmap or vector; saving a scanned piece of line art as PDF does not convert it to scalable line art.
        \item If you generate graphics using a \TeX\ package, please be sure to provide a PDF of the manuscript.  In the production process, \TeX-generated graphics will eventually be converted to more conventional graphics so the \textit{Mag} can be delivered in e-reader formats.  We prefer graphics produced by draw programs so use \TeX-generated art as a last resort.
        \item For photos of contributing authors, we prefer photos that are not cropped tight to the author's profile, so that production staff can crop the head shot to an equal height and width.  If possible, avoid photographs that have excess shadows or glare.
    \end{enumerate}



    \section{Theorems, definitions, proofs, and all that}

    Following the defaults of the \texttt{amsthm} package, styling is provided for \texttt{theorem}, \texttt{definition}, and \texttt{remark} styles, although the latter two use the same styling.


    \begin{theorem}[Pythagorean Theorem]
        Theorems, lemmas, axioms, and the like are stylized using italicized text. These environments can be numbered or unnumbered, at the author's discretion.
    \end{theorem}

    \begin{proof}
        Proofs set in roman (upright) text, and conclude with an ``end of proof'' (q.e.d.) symbol that is set automatically when you end the proof environment.  When the proof ends with an equation or other non-text element, you need to add \verb~\qedhere~ to the element to set the end of proof symbol; see the \texttt{amsthm} package documentation for more details.
    \end{proof}

    \begin{definition}[Secant Line]
        Definitions, remarks, and notation are stylized as roman text.  They are typically unnumbered, but there are no hard-and-fast rules about numbering.
    \end{definition}

    \begin{remark}
        Remarks stylize the same as definitions.
    \end{remark}

    \begin{thebibliography}{3}

        \bibitem{example1}
        Leader, S. (1986). What is a differential? A new answer from teh generalized Riemann integral. {\it Amer. Math. Monthly.\/} 93(5): 348--356.

        \bibitem{example2}
        Steeb, W.-H. (1996). \textit{Continuous Symmetries, Lie Algebras, Differential Equations and Computer Algebra.\/} River Edge, NJ: World Scientific Publishing.  \href{http://dx.doi.org/10.1142/3309}{\url{http://dx.doi.org/10.1142/3309}}

        \bibitem{example3}
        Titchmarsh, E. C. (1986). {\it The Theory of the Riemann Zeta-Function.\/} 2nd. ed. Edited and with a preface by D. R. Heath-Brown. New York: The Clarendon Press, Oxford Univ. Press.

    \end{thebibliography}

\end{document}



%%! Author = Nilo
%%! Date = 30/08/2019
%% Preamble
%\documentclass[a4paper,10pt,oneside]{book}
%\usepackage{amssymb}
\usepackage{amsmath}
\usepackage{amsthm}
\usepackage{enumitem}
\usepackage{amsfonts}
\usepackage{xcolor}
\usepackage{tcolorbox}
\usepackage{float}
\usepackage{graphicx}
\usepackage{hyperref}

\colorlet{shadecolor}{gray!20}

\newcommand{\Date}{--/9-'19}
%
\newcommand{\questionBox}[1]{
\begin{tcolorbox}[width=\textwidth,
colback={shadecolor},
title={Question:},colbacktitle=white,coltitle=black]
    #1
\end{tcolorbox}
}
%
\newcommand{\answerBox}[1]{
\begin{tcolorbox}[width=\textwidth,
title={Answer:},colbacktitle=white,coltitle=black]
    #1
\end{tcolorbox}
}
%
%\title{Number Theory Problem Sets}
%\author{Nilo de Roock}
%\date{\Date}
%
%\begin{document}
%
%    \maketitle
%    %\newpage+
%    \tableofcontents{}
%    \newpage
%    \thispagestyle{empty}
%
%%    \chapter{Complex Analysis}

\section{Numbers}

\paragraph{Algebra}

\paragraph{Conjugate}

\paragraph{Polar Representation}

\paragraph{Euler's Formula}

\paragraph{De Moivre's Theorem}

\paragraph{n-th Roots of Unity}

\paragraph{Complex Exponents}

\paragraph{Complex Logarithm}


\section{Differentiation}

\paragraph{Limits}

\paragraph{Continuity}

\paragraph{Derivatives}

\paragraph{Rules for Differentiation}

\paragraph{Cauchy-Riemann Equations}

\paragraph{Analytic Functions}

\paragraph{Harmonic Functions}



%%    \chapter{Glossary}


\begin{definition}[Curve (II.1.1)]
    \label{sec:Curve}
    A \textbf{curve} is a continuous map
    $$\alpha: [a,b] \rightarrow \mathbb{C}, a < b$$
    from a compact real interval into the complex plane. We call $\alpha(a)$ the starting point,
    and $\alpha(b)$ the end point of $\alpha$.
\end{definition}


\begin{definition}[Smooth Curve (II.1.2)]
    \label{sec:SmoothCurve}
    A \hyperref[sec:Curve]{curve} is called \textbf{smooth}, if it is continuously differentiable.
\end{definition}


\begin{definition}[Piecewise Smooth Curve (II.1.3)]
    \label{sec:PiecewiseSmoothCurve}
    Let\\
    TBD II.1.3
\end{definition}


\begin{definition}[Contour Integral (II.1.4)]
    \label{sec:ContourIntegral}
    Let
    $$\alpha : [a, b] \rightarrow \mathbb{C}$$
    be \hyperref[sec:SmoothCurve]{a smooth curve} and
    $$f: D \rightarrow \mathbb{C}, D \subset \mathbb{C},$$
    a continuous funtion, whose domain of definition contains the image of the curve $\alpha$,
    i.e. $D \supset \alpha([a,b]).$ Then one defines
    $$ \int_\alpha f:= \int_\alpha f(\zeta) d\zeta := \int_a^b f(\alpha(t))\alpha'(t)dt,$$
    and calls this complex number the line integral or \textbf{contour integral} of $f$ along $\alpha$.
    By the arc length of a smooth curve we mean
    $$l(\alpha):=\int_a^b |\alpha'(t)|dt.$$
\end{definition}


\begin{definition}[Properties Contour Integral (II.1.5)]
    \label{sec:PropContourIntegral}
    Let\\
    TBD
\end{definition}


\begin{definition}[Arcwise Connected Set (II.2.1)]
    \label{sec:ArcwiseConnected}
    Let\\
    TBD
\end{definition}


\begin{definition}[Domain (II.2.3)]
    \label{sec:Domain}
    By a \textbf{domain} we understand an \hyperref[sec:ArcwiseConnected]{arcwise connected} non-empty open set $D \subset \mathbb{C}$.
\end{definition}


\begin{definition}[A Star-shaped Domain (II.2.6)]
    \label{sec:StarDomain}
    A \textbf{star-shaped domain} is an open set $D \subset \mathbb{C}$ with the following property:
    There is a point $z_\star \in D$ such that for each point $z \in D$ the whole line segment joining $z_\star$ and $z$ is
    contained in $D$:
    $$\{ z_\star + t(z - z_\star) ; t \in [0, 1] \} \subset D .$$
    The point $z_\star$ is not uniquely determined, and is called a (possible) star center.
\end{definition}


\begin{definition}[Elementary Domain (II.2.8)]
    \label{sec:ElemDomain}
    A domain $D \subset \mathbb{C}$ is called an \textbf{elementary domain}, if
    any analytic function defined on $D$ has a primitive in $D$.
\end{definition}


\begin{definition}[Entire Function (II.3.6)]
    \label{sec:EntireFunction}
    An analytic function $f : \mathbb{C} \rightarrow \mathbb{C}$ is said to be \textbf{entire}.
\end{definition}
%
%    \chapter{THEORY}
%    
\section{Euclid}

A prime number is a positive integer not equal to $1$ which is divisible only by $1$ and itself.\newline

The sequence of prime numbers starts with
$$2,3,5,7,11,13,17,19,23,29,31,37,41,43,47,53,59,61,67,71,73,79,83,89,97, \ldots$$

The Fundamental Theorem of Arithmetic claims that every integer greater than 1 either is a prime number itself or can be
represented as the product of prime numbers and that, moreover, this representation is unique, up to (except for) the
order of the factors. So, if
$$p_1=2, p_2=3, p_3=5, \ldots$$
and in general $p_n$ is equal to the $n$-th prime number and $a>1$ is a positive integer, then
$$a=p_{\alpha_1}^{b_1} \ldots p_{\alpha_\nu}^{b_\nu}$$
where $\nu \geq 1, 1 \leq \alpha_1 < \alpha_2 \ldots < \alpha_\nu$ and the sum of the exponents $b_i \geq 1$, is the
unique representation of $a$.


%
%
%    \chapter{PROBLEM SETS}
%%    \section{TMA: Fundamental Theorem of Arithmetic}

\subsection[Problem 1]{Problem 1: Apo 1.17}

\subsection[Problem 2]{Problem 1: Apo 1.19}

\subsection[Problem 3]{Problem 1: Apo 1.20}

\subsection[Problem 4]{Problem 1: Apo 1.23}

\subsection[Problem 5]{Problem 1: Apo 1.25}
Show that: if $(a, b) = 1$ there exist $x >0$ and $y > 0$ such that $ax - by = 1$.

\subsection[Problem 6]{Problem 1: Apo 1.26}

\subsection[Problem 7]{Problem 1: Apo 1.27}

\subsection[Problem 8]{Problem 1: Apo 1.28}

\subsection[Problem 9]{Problem 1: Apo 1.29}

\subsection[Problem 10]{Problem 1: Apo 1.30}
%%    \newpage
%
%    \section{TMA: Arithmetical Functions and Dirichlet Multiplication}

\subsection[Problem 1]{Problem 1: Apo 2.11}

$$
g(x)=\sum _{n=1}^{\lfloor x\rfloor } f\left(\frac{x}{n}\right)
$$
$$
g(\text{x$\_$})\text{:=}\sum _{n=1}^{\lfloor x\rfloor } f\left(\frac{x}{n}\right)
$$

\subsection[Problem 2]{Problem 2: Apo 2.12}

\subsection[Problem 3]{Problem 3: Apo 2.13}

\subsection[Problem 4]{Problem 4: Apo 2.15}

\subsection[Problem 5]{Problem 5: Apo 2.18}

\subsection[Problem 6]{Problem 6: Apo 2.19}

\subsection[Problem 7]{Problem 7: Apo 2.21}

\subsection[Problem 8]{Problem 8: Apo 2.22}

\subsection[Problem 9]{Problem 9: Apo 2.32}

\subsection[Problem 10]{Problem 10: Apo 2.32}
%    \newpage
%
%%    \section{TMA: Averages of Arithmetical Functions}

\subsection[Problem 1]{Problem 1: Apo 3.1}
Use Euler summation to deduce for $x \geq 2:$
\begin{itemize}
    \item[a)] $\sum_{n \leqslant x} \frac{\log{n}}{n} = \frac{1}{2} \log^2{x} + A + \mathcal{O}(\frac{\log{x}}{x})$
    \item[b)] $\sum_{2 \leqslant n \leqslant x} \frac{1}{n \log{n}} = \log{\log{x}} + B + \mathcal{O}(\frac{1}{x \log{x}})$
\end{itemize}

\subsection[Problem 2]{Problem 2: Apo 3.2}

\subsection[Problem 3]{Problem 3: Apo 3.3}

\subsection[Problem 4]{Problem 4: Apo 3.4}

\subsection[Problem 5]{Problem 5: Apo 3.5}

\subsection[Problem 6]{Problem 6: Apo 3.6}

\subsection[Problem 7]{Problem 7: Apo 3.7}

\subsection[Problem 8]{Problem 8: Apo 3.8}

\subsection[Problem 9]{Problem 9: Apo 3.9}

\subsection[Problem 10]{Problem 10: Apo 3.10}

\subsection[Problem 11]{Problem 11: Apo 3.11}

\subsection[Problem 12]{Problem 12: Apo 3.12}

\subsection[Problem 13]{Problem 13: Apo 3.13}

\subsection[Problem 14]{Problem 14: Apo 3.14}

\subsection[Problem 15]{Problem 15: Apo 3.15}

\subsection[Problem 16]{Problem 16: Apo 3.16}

\subsection[Problem 17]{Problem 17: Apo 3.17}

\subsection[Problem 18]{Problem 18: Apo 3.18}

\subsection[Problem 19]{Problem 19: Apo 3.19}

\subsection[Problem 20]{Problem 20: Apo 3.20}

\subsection[Problem 21]{Problem 21: Apo 3.21}

\subsection[Problem 22]{Problem 22: Apo 3.22}

\subsection[Problem 23]{Problem 23: Apo 3.23}

\subsection[Problem 24]{Problem 24: Apo 3.24}

\subsection[Problem 25]{Problem 25: Apo 3.25}

\subsection[Problem 26]{Problem 26: Apo 3.26}

%%    \newpage
%%    \section{TMA: Some Elementary Theorems on the Distribution of Prime Numbers}

\subsection[Problem 1]{Problem 1: Apo 4.1}
Let  $S = \{1, 5, 9, 13, 17, \ldots \}$ denote the set of all positive integers of the form $4n + 1$.
An element $p$ of $S$  is called an S-prime if $p > 1$ and if the only divisors of p, among the elements of S,
are $1$ and $p$.  (For example, $49$ is an S-prime .) An element $n > 1$ in S which is not an S-prime is called an
S-composite.
\begin{itemize}
    \item[a)] Prove that every S-composite is a product of S-primes.
    \item[b)] Find the smallest S-composite that can be expressed in  more than one way as a product of S-primes.
\end{itemize}

\subsection[Problem 2]{Problem 2: Apo 4.2}
Consider $T = \{1, 7, 11, 13, 17, 19, 23, 29\}$.
\begin{itemize}
    \item[a)] For each prime $p$ in the interval $30 < p < 100$ determine a pair of integers $m, n$, where $m > 0$
    and $n \in T$, such that $p = 30m +  n$.
    \item[b)] Prove or disprove: "Every prime $p > 5$ can be expressed in the form $30m + n$, $m, n$, where $m > 0$
    and $n \in T$".
\end{itemize}

\subsection[Problem 3]{Problem 3: Apo 4.3}
.

\subsection[Problem 4]{Problem 4: Apo 4.4}

\subsection[Problem 5]{Problem 5: Apo 4.5}

\subsection[Problem 6]{Problem 6: Apo 4.6}

\subsection[Problem 7]{Problem 7: Apo 4.7}

\subsection[Problem 8]{Problem 8: Apo 4.8}

\subsection[Problem 9]{Problem 9: Apo 4.9}

\subsection[Problem 10]{Problem 10: Apo 4.10}

\subsection[Problem 11]{Problem 11: Apo 4.11}

\subsection[Problem 12]{Problem 12: Apo 4.12}

\subsection[Problem 13]{Problem 13: Apo 4.13}

\subsection[Problem 14]{Problem 14: Apo 4.14}

\subsection[Problem 15]{Problem 10: Apo 4.15}

\subsection[Problem 16]{Problem 10: Apo 4.16}

\subsection[Problem 17]{Problem 10: Apo 4.17}

\subsection[Problem 18]{Problem 10: Apo 4.18}

\subsection[Problem 19]{Problem 10: Apo 4.19}

\subsection[Problem 20]{Problem 10: Apo 4.20}

\subsection[Problem 21]{Problem 10: Apo 4.21}

\subsection[Problem 22]{Problem 10: Apo 4.22}

\subsection[Problem 23]{Problem 10: Apo 4.23}

\subsection[Problem 24]{Problem 10: Apo 4.24}

\subsection[Problem 25]{Problem 10: Apo 4.25}

\subsection[Problem 26]{Problem 10: Apo 4.26}

\subsection[Problem 27]{Problem 10: Apo 4.27}

\subsection[Problem 28]{Problem 10: Apo 4.28}

\subsection[Problem 29]{Problem 10: Apo 4.29}

\subsection[Problem 30]{Problem 10: Apo 4.30}
%%    \newpage
%%    \section{TMA: Congruences}

\subsection[Problem 1]{Problem 1: Apo 5.1}

\subsection[Problem 2]{Problem 2: Apo 5.2}

\subsection[Problem 3]{Problem 3: Apo 5.3}

\subsection[Problem 4]{Problem 4: Apo 5.4}

\subsection[Problem 5]{Problem 5: Apo 5.5}

\subsection[Problem 6]{Problem 6: Apo 5.6}

\subsection[Problem 7]{Problem 7: Apo 5.7}

\subsection[Problem 8]{Problem 8: Apo 5.8}

\subsection[Problem 9]{Problem 9: Apo 5.9}

\subsection[Problem 10]{Problem 10: Apo 5.10}
%%    \newpage
%%    \section{TMA: Finite Abelian Groups and their Characters}

\subsection[Problem 1]{Problem 1: Apo 6.3}

\subsection[Problem 2]{Problem 2: Apo 6.9}

\subsection[Problem 3]{Problem 3: Apo 6.14}

\subsection[Problem 4]{Problem 4: Apo 6.16}

\subsection[Problem 5]{Problem 5: Apo 6.17}
%%    \newpage
%%    \section{TMA: Dirichlet's Theorem on Primes in Arithmetical Progressions}

\subsection[Problem 1]{Problem 1: Apo 7.7}

\subsection[Problem 2]{Problem 2: Apo 7.8}

%%    \newpage
%%    \section{TMA: Periodic Arithmetical Functions and Gauss Sums}

\subsection[Problem 1]{Problem 1: Apo 8.1}

\subsection[Problem 2]{Problem 2: Apo 8.3}

\subsection[Problem 3]{Problem 3: Apo 8.7}

\subsection[Problem 4]{Problem 4: Apo 8.9}

\subsection[Problem 5]{Problem 5: Apo 8.11}
%%    \newpage
%%    \section{TMA: Quadratic Residues and the Quadratic Reciprocity Law}

\subsection[Problem 1]{Problem 1: Apo 9.1}

\subsection[Problem 2]{Problem 2: Apo 9.2}

\subsection[Problem 3]{Problem 3: Apo 9.3}

\subsection[Problem 4]{Problem 4: Apo 9.4}

\subsection[Problem 5]{Problem 5: Apo 9.5}

\subsection[Problem 6]{Problem 6: Apo 9.6}

\subsection[Problem 7]{Problem 7: Apo 9.7}

\subsection[Problem 8]{Problem 8: Apo 9.8}

\subsection[Problem 9]{Problem 9: Apo 9.9}

\subsection[Problem 10]{Problem 10: Apo 9.10}
%%    \newpage
%%    \section{TMA: Primitive Roots}

\subsection[Problem 1]{Problem 1: Apo 10.4}

\subsection[Problem 2]{Problem 2: Apo 10.5}

\subsection[Problem 3]{Problem 3: Apo 10.7}

\subsection[Problem 4]{Problem 4: Apo 10.10}

\subsection[Problem 5]{Problem 5: Apo 10.16}
%%    \newpage
%%    \section{TMA: Dirichlet Series and Euler Products}

\subsection[Problem 1]{Problem 1: Apo 11.5}

\subsection[Problem 2]{Problem 2: Apo 11.6}

\subsection[Problem 3]{Problem 3: Apo 11.7}

\subsection[Problem 4]{Problem 4: Apo 11.8}

\subsection[Problem 5]{Problem 5: Apo 11.9}

\subsection[Problem 6]{Problem 6: Apo 11.10}

\subsection[Problem 7]{Problem 7: Apo 11.11}

\subsection[Problem 8]{Problem 8: Apo 11.12}

\subsection[Problem 9]{Problem 9: Apo 11.15}

\subsection[Problem 10]{Problem 10: Apo 11.17}
%%    \newpage
%%    \section{TMA: Zeta- and L-Functions}

\subsection[Problem 1]{Problem 1: Apo 12.1}

\subsection[Problem 2]{Problem 2: Apo 12.2}

\subsection[Problem 3]{Problem 3: Apo 12.6}

\subsection[Problem 4]{Problem 4: Apo 12.7}

\subsection[Problem 5]{Problem 5: Apo 12.10}
%%    \newpage
%%    \section{TMA: Analytic Proof of the PNT}

\subsection[Problem 1]{Problem 1: Apo 13.2}

\subsection[Problem 2]{Problem 2: Apo 13.4}

\subsection[Problem 2]{Problem 2: Apo 13.7}

%%    \newpage
%%    \section{TMA: Partitions}

\subsection[Problem 1]{Problem 1: Apo 14.1}

\subsection[Problem 2]{Problem 2: Apo 14.2}

\subsection[Problem 2]{Problem 2: Apo 14.4}

%
%\end{document}