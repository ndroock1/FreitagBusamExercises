\documentclass{article} 
\usepackage{mathmag}

\usepackage{amsmath,amsthm}     
\usepackage{graphicx}     
\usepackage{hyperref} 
\usepackage{url}
\usepackage{amsfonts} 

\usepackage{multicol}


% NOTE mathmag.sty calls the text fonts. For this template we are using times.sty 
% from the standard LaTeX distribution.

%% IF YOU HAVE FONTS INSTALLED you can use these math fonts to more
%% closely approximate the final product.
%\usepackage{mtpro2}
%\usepackage{mathtime}

\theoremstyle{theorem}
\newtheorem{theorem}{Theorem}

\theoremstyle{definition}
\newtheorem*{definition}{Definition}
\newtheorem*{remark}{Remark}

\allowdisplaybreaks

\makeatletter
\@addtoreset{footnote}{page}
\makeatother

%%%%%%%%%%%%%%%%%%%%%%%%%%%%%%%%%%%%%%%%%%%%%%%%%%
\begin{document}


\title{The Math Mag Article Template}

\author{Author Name\\               %%%% Leave ALL of these as is in your initial submission
\scriptsize affiliation line 1\\    %%%% to allow for double blind reviewing.
affiliation line 2\\                %%%% They should be filled in when you are submitting
email address}                      %%%% your final manuscript.

\maketitle

\noindent  Mathematics Magazine aims to provide lively and appealing mathematical exposition. The Magazine is not a research journal, so the terse style appropriate for such a journal (lemma-theorem-proof-corollary) is not appropriate for the Magazine. Articles should include examples, applications, historical background, and illustrations, where appropriate. They should be attractive and accessible to undergraduates and would, ideally, be helpful in supplementing undergraduate courses or in stimulating student investigations. Manuscripts on history are especially welcome, as are those showing relationships among various branches of mathematics and between mathematics and other disciplines.

Submissions of articles are required via the Mathematics Magazine's Editorial Manager System. The name(s) of the author(s) should not appear in the file. Initial submissions in pdf or LaTeX form can be sent to the editor at \href{http://www.editorialmanager.com/mathmag/}{\url{www.editorialmanager.com/mathmag/}}. 
The Editorial Manager System will cue the author for all required information concerning the paper. Questions concerning submission of papers can be addressed to the editor at \href{mathmag@maa.org}{\url{mathmag@maa.org}}.

\section{Mathematics Magazine style}

The \textit{Mathematics Magazine} style incorporates the following \LaTeX\ packages.  These styles should \textit{not} be included in the document header.
\begin{itemize}
\item times
\item pifont
\item graphicx
\item color
\item AMS styles: amsmath, amsthm, amsfonts, amssymb
\item url
\end{itemize}
Use of other \LaTeX\ packages should be minimized as much as possible. Math notation, like $c = \sqrt{a^2 +b^2}$, can be left in \TeX's default Computer Modern typefaces for manuscript preparation; or, if you have the appropriate fonts installed, the \texttt{mathtime} or \texttt{mtpro} packages may be used, which will better approximate the finished article.

Web links can be embedded using the \verb~\url{...}~ command, which will result in something like \url{http://www.maa.org}.  These links will be active and stylized in the online publication.

\section{First-level section heading}

Section headings use an initial capital letter on the first word, with subsequent words lowercase.  In general, the style of the journal is to leave all section headings unnumbered.  Consult the journal editor if you wish to depart from this and other conventions.

\subsection{Second-level heading}

The same goes for second-level headings.  It is not necessary to add font commands to make the math within heads bold and sans serif; this change will occur automatically when the production style is applied.

\section{Graphics and tables}

Table for  \textit{Math Mag} should be set in an ``open" style: rules above and below the heading and a rule to end the table.  Note the use of \verb~\abrule~ and \verb~\brule~ to improve spacing in the table.

\begin{table}[h]
\begin{center}
\begin{tabular}{ccc}
\hline
Under  & $\pi(x) = \#\{\text{primes} \le x\}=$ &   $=\text{Li}(x)\pm$  Error\abrule\\
\hline
$500000$  &  41556 &  $41606.4 - 50.4$ \abrule \\
$1000000$ &  78501 &  $79627.5 - 126.5$\brule \\
$1500000$ & 114112 & $114263.1 - 151.1$\brule \\
$2000000$ & 148883 & $149054.8 - 171.8$\brule \\
$2500000$ & 183016 & $183245.0 - 229.0$\brule \\
$3000000$ & 216745 & $216970.6 - 225.6 $\brule \\
\hline
\end{tabular}
\end{center}
\caption{Sample table}
\end{table}


Figures for  \textit{Math Mag} can be submitted as either color or black \& white graphics.  Generally, color graphics will be used for the online publication, and converted to black \& white images for the print journal.  We recommend using whatever graphics program you are most comfortable with, so long as the submitted graphic is provided as a separate file using a standard file format.

For best results, please follow the following guidelines:
\begin{enumerate}
\item Bitmapped file formats---preferably TIFF or JPEG, but not BMP---are appropriate for photographs, using a resolution of at least 300 dpi at the final scaled size of the image.
\item Line art will reproduce best if provided in vector form, preferably EPS. The thinnest line weight should be .5 pt.  Labels on a figure should be 9 pt in the same font style (italic, bold, etc.) as in the text.
\item Alternatively, both photographs and line art can be provided as PDF files.  Note that creating a PDF does not affect whether the graphic is a bitmap or vector; saving a scanned piece of line art as PDF does not convert it to scalable line art.
\item If you generate graphics using a \TeX\ package, please be sure to provide a PDF of the manuscript.  In the production process, \TeX-generated graphics will eventually be converted to more conventional graphics so the \textit{Mag} can be delivered in e-reader formats.  We prefer graphics produced by draw programs so use \TeX-generated art as a last resort.
\item For photos of contributing authors, we prefer photos that are not cropped tight to the author's profile, so that production staff can crop the head shot to an equal height and width.  If possible, avoid photographs that have excess shadows or glare.
\end{enumerate}



\section{Theorems, definitions, proofs, and all that}

Following the defaults of the \texttt{amsthm} package, styling is provided for \texttt{theorem}, \texttt{definition}, and \texttt{remark} styles, although the latter two use the same styling.


\begin{theorem}[Pythagorean Theorem]
Theorems, lemmas, axioms, and the like are stylized using italicized text. These environments can be numbered or unnumbered, at the author's discretion.
\end{theorem}

\begin{proof}
Proofs set in roman (upright) text, and conclude with an ``end of proof'' (q.e.d.) symbol that is set automatically when you end the proof environment.  When the proof ends with an equation or other non-text element, you need to add \verb~\qedhere~ to the element to set the end of proof symbol; see the \texttt{amsthm} package documentation for more details.
\end{proof}

\begin{definition}[Secant Line]
Definitions, remarks, and notation are stylized as roman text.  They are typically unnumbered, but there are no hard-and-fast rules about numbering.
\end{definition}

\begin{remark}
Remarks stylize the same as definitions.
\end{remark}

\begin{thebibliography}{3}

\bibitem{example1}
Leader, S. (1986). What is a differential? A new answer from teh generalized Riemann integral. {\it Amer. Math. Monthly.\/} 93(5): 348--356.

\bibitem{example2}
Steeb, W.-H. (1996). \textit{Continuous Symmetries, Lie Algebras, Differential Equations and Computer Algebra.\/} River Edge, NJ: World Scientific Publishing.  \href{http://dx.doi.org/10.1142/3309}{\url{http://dx.doi.org/10.1142/3309}}


\bibitem{example3}
Titchmarsh, E. C. (1986). {\it The Theory of the Riemann Zeta-Function.\/} 2nd. ed. Edited and with a preface by D. R. Heath-Brown. New York: The Clarendon Press, Oxford Univ. Press.

\end{thebibliography}

\end{document}
